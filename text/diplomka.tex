\documentclass[12pt]{article}
\usepackage[master,figures]{updiplom}
\usepackage[utf8]{inputenc}
% \usepackage{epsfig}
\usepackage{url}
% \DeclareUrlCommand\url{\def\UrlLeft{<}\def\UrlRight{>} \urlstyle{tt}}
\usepackage{enumitem}

\usepackage{hyperref}
\hypersetup{backref,
linktocpage,
colorlinks=true,
linkcolor=blue,
citecolor=blue,
urlcolor=blue}

\usepackage{minted}
\usemintedstyle{autumn}
\usepackage{placeins}

\usepackage{framed}

\newcommand{\registered}{\textsuperscript{\textregistered}}

\title{Implementace expertního systému v jazyce Common Lisp}
\author{Jakub Kaláb}
\year{2013}
\date{31.12.2013}
\annotation{
Expertní systémy mají v~praxi bohaté využití. Jejich smyslem je asistovat
expertovi na danou problematiku, či jej plně nahradit. V~příloze bakalářské
práce implementuji prázdný expertní systém s~dopředným řetězením inspirovaný
systémem CLIPS jako knihovnu v~programovacím jazyku Common Lisp tak, aby jej
bylo možno plně integrovat do dalších programů.
}
\thanks{Děkuji Mgr. Martinu Dostálovi, Ph.D. za vedení této diplomové práce.}

%%%%%%%%%%%%%%%%%%%%%%%%%%%%%%%%%%%%%%%%%%%%%%%%%%%%%%%%%%%%%%%%%%%%%%%%%%%%%%%%
\begin{document}

\maketitle
\renewcommand\listoflistingscaption{Seznam příkladů}
\renewcommand\listingscaption{Příklad}
\listoflistings
\newminted{cl}{samepage,linenos,frame=lines}
\newmint{cl}{}

\nocite{bakalarka}
\nocite{introduction}
\nocite{paradigms}
\nocite{doorenbos}
\nocite{clips}
\nocite{expert-system}
\nocite{rete}

\clearpage
%%%%%%%%%%%%%%%%%%%%%%%%%%%%%%%%%%%%%%%%%%%%%%%%%%%%%%%%%%%%%%%%%%%%%%%%%%%%%%%%
\section{Úvod}

Pojem expertního systému spadá do oblasti umělé inteligence. Jde o počítačový
systém, který simuluje rozhodování experta nad zvolenou problémovou doménou.
Expertní systém může experta zcela nahradit, nebo mu při rozhodování asistovat.

Ve své bakalářské práci \cite{bakalarka} jsem implementoval základní knihovnu pro tvorbu
expertních systémů (tzv. prázdný expertní systém) s~dopředným řetězením v~jazyce
Common Lisp. Cílem této práce je knihovnu rozšířit o následující:
\begin{itemize}
  \item syntaktický režim pro zajištění přiměřené kompatibility se systémem
    CLIPS,
  \item možnost vrácení provedených změn včetně odvozovacích kroků,
  \item podpora pro ladění s jednoduchým grafickým uživatelským rozhraním pro
    prostředí LispWorks\texttrademark,
  \item rozšíření odvozovacího aparátu o základní zpětné řetězení.
\end{itemize}

Jazyk Common Lisp\footnote{\url{http://en.wikipedia.org/wiki/Common\_Lisp}}
(případně jiné dialekty Lispu) je častou volbou pro implementaci umělé
inteligence díky svým schopnostem v oblasti symbolických výpočtů (manipulace
symbolických výrazů), na nichž řešení těchto problémů často staví. Navíc jde
o~velmi vysokoúrovňový, dynamicky typovaný jazyk, díky čemuž je programový kód
expresivní, snadno pochopitelný a tudíž jednoduše rozšiřitelný.

Synax systému CLIPS\footnote{\url{http://clipsrules.sourceforge.net}} byla
zvolena proto, že jde o reálně používaný
systém\footnote{\url{http://clipsrules.sourceforge.net/FAQ.html\#Q6}}, jehož
syntax je Lispu velmi blízká, takže není těžké ji v~Lispu napodobit.

Přestože běžnou praxí je začínat diplomovou práci teoretickou částí, definovat
jednotlivé pojmy a principy a ty poté v praktické části uplatnit, rozhodl jsem
se postupovat opačně, tedy začít práci praktickou částí. Domnívám se totiž (také
na základě zkušeností nabytých při vypracování bakalářské práce), že je podstatně
snazší, minimálně v řešené problematice, pochopit příklady bez detailní znalosti
teorie, než snažit se pochopit teorii bez příkladů, na nichž si lze popisované
pojmy a principy představit. V praktické části tedy uvedu jen minimální množtví
teorie nutné pro pochopení práce s knihovnou, načež se k ní v teoretické části
textu vrátím, pojmy zadefinuji přesně a rozšířím o souvislosti. V tuto chvíli už
si bude čtenář schopen představit, jaké problémy lze pomocí expertního systému
řešit a jak takový expertní systém v praxi vypadá.

\begin{framed}
  Pro popsaní některé teorie (symbolické výpočty, ...) je navíc užitečná (ne-li
  potřebná) představa, jak expertní systém funguje uvnitř.
\end{framed}

\begin{framed}
  Kdo už to dělal, čím se tohle liší
\end{framed}


\clearpage
\section{Praktická část}

Tato sekce popisuje knihovnu ExiL\footnotemark, která je výsledkem této práce.
Nejprve popíšu její instalaci a prerekvizity nutné k jejímu používání. Pak v
uživatelské příručce představím její možnosti a typickou strukturu programu,
který ji využívá.  Poté v~referenční příručce projdu všechny možnosti, které
knihovna poskytuje.  Načež v části věnované implementaci popíšu architekturu
jejího zdrojového kódu a zmíním zajímavé části kódu implementující jednotlivá
rozšíření.  Nakonec uvedu několik větších příkladů použití knihovny a~rozeberu
několik dalších možných rozšíření a co by obnášela z pohledu implementace.

\footnotetext{původcem názvu knihovny je Zdenek Eichler}

\subsection{Common Lisp}
\begin{framed}
\begin{itemize}
  \item základní znalosti lispu nutné pro používání knihovny
  \begin{itemize}
    \item package, export, import, shadow
    \item seznam, atom, car, cdr, plist
    \item symbol, klíč
    \item prefixová syntax, S-expressions
    \item funkce, makro
    \item načítání souborů.
    \item quotování (u funkčních alternativ, ale ty se stejně používají spíš z jiného
      kódu, který knihovnu volá - tudíž uživatel evidentně lisp zná)
  \end{itemize}
  \item odkaz na practical common lisp, clhs
\end{itemize}
\end{framed}

%%%%%%%%%%%%%%%%%%%%%%%%%%%%%%%%%%%%%%%%%%%%%%%%%%%%%%%%%%%%%%%%%%%%%%%%%%%%%%%%
\subsection{Instalace}
\subsubsection{Získání zdrojového kódu}
Zdrojový kód knihovny ExiL je přiložen k~této diplomové práci a~lze jej také získat
zklonováním\footnote{\url{http://git-scm.com/docs/git-clone}}
gitového\footnote{\url{http://git-scm.com/}} repozitáře na adrese
\verb|git@github.com:Incanus3/ExiL.git|.
Kód knihovny se nachází v podadresáři \verb|src|, ten budu dále nazývat kořenovým
adresářem knihovny či projektu.

%%%%%%%%%%%%%%%%%%%%%%%%%%%%%%%%%%%%%%%%%%%%%%%%%%%%%%%%%%%%%%%%%%%%%%%%%%%%%%%%
\subsubsection{Prerekvizity}
Pro práci s knihovnou ExiL potřebujeme lispový
interpreter\footnote{lispové interpretery jsou většinou zároveň
kompilátory, %\footnotemark[11],
označením interpreter tedy budu nazývat obojí}, vývojové prostředí (s
interpreterem bychom si ve skutečnosti vystačili, ale přímá práce s~ním není
většinou příliš pohodlná) a knihovny umožňující dávkové načtení celého projektu
včetně závislostí.
%\footnotetext[11]{\url{http://en.wikipedia.org/wiki/Compiler}}
%\addtocounter{footnote}{1}\addtocounter{Hfootnote}{1}

Knihovnu jsem vyvíjel v prostředí
SLIME\footnote{\url{http://www.common-lisp.net/project/slime/}}, což je plugin
pro textový editor GNU Emacs\footnote{\url{http://www.gnu.org/software/emacs/}}
(poskytující mimo jiné pomůcky pro editaci lispového zdrojového kódu,
REPL\footnote{\url{http://en.wikipedia.org/wiki/Read-eval-print\_loop}} a
debugger) %\footnote{\url{http://en.wikipedia.org/wiki/Debugger}}
s interpreterem SBCL\footnote{\url{http://www.sbcl.org/}} a tuto kombinaci
mohu vřele doporučit. V operačním systému Debian GNU Linux, který jsem pro vývoj
použil, lze Emacs, SLIME i SBCL nainstalovat z výchozího repozitáře a aktivovat
úpravou inicializačního souboru
Emacsu\footnote{\url{http://www.common-lisp.net/projects/slime/doc/html/Installation.html}}.
Prostředí poté můžeme v~Emacsu spustit voláním příkazu \verb|slime|
(\verb|<Alt+X>slime<ENTER>|). Při prvním spuštění se kód prostředí kompiluje, což
může chvíli trvat, pak už se v editoru otevře
buffer\footnote{\url{http://www.gnu.org/software/emacs/manual/html\_node/emacs/Buffers.html}}
s~lispovým REPLem.

Knihovnu jsem testoval také ve vývojovém prostředí
LispWorks\registered\footnote{\url{http://www.lispworks.com/}} Personal Edition
6.1, pro které jsem také vytvořil minimalistické grafické uživatelské rozhraní.
Součástí prostředí LispWorks je i lispový interpret. Prostředí můžeme získat na
adrese \url{http://www.lispworks.com/downloads/index.html} a nainstalovat podle
návodu, který se zobrazí po vyplnění formuláře.

Pro efektivní načtení knihovny včetně závislostí potřebujeme ještě dvě knihovny:
\clearpage
\begin{itemize}
  \item ASDF\footnote{\url{http://common-lisp.net/project/asdf}} je knihovna
    umožňující snadnou definici struktury projektu a jeho dávkové načtení,
  \item quicklisp\footnote{\url{http://www.quicklisp.org/beta/}} staví na knihovně
    ASDF a umožňuje pohodlně stáhnout a načíst knihovny třetích stran z internetové
    databáze.
\end{itemize}
Knihovna ASDF je součástí instalace interpreteru SBCL i prostředí LispWorks.
Knihovnu quicklisp jsem k projektu přiložil a pokud není součástí prostředí, je
automaticky načtena před načtením ExiLu.
%%%%%%%%%%%%%%%%%%%%%%%%%%%%%%%%%%%%%%%%%%%%%%%%%%%%%%%%%%%%%%%%%%%%%%%%%%%%%%%%
\subsubsection{Načtení knihovny}
V prostředí SLIME načteme knihovnu načtením souboru \verb|load.lisp| z kořenového
adresáře knihovny, tedy zadáním \cl|(load "cesta/k/projektu/src/load.lisp")|
v~REPLu). Tento soubor nejprve načte knihovnu \verb|quicklisp|, je-li potřeba,
a s její pomocí poté načte celý projekt ExiL včetně závislostí. Nakonec soubor
definuje výchozí prostředí, viz kapitola \ref{multiple environments}

V prostředí LispWorks načítání pomocí knihovny \verb|quicklisp| nefunguje správně,
knihovnu je proto třeba načítat načtením souboru \verb|load-manual.lisp| (opět
z kořenového adresáře projektu). Načíst můžeme opět voláním \verb|load| v~REPLu,
nebo vybráním položky \verb|Load...| v nabídce \verb|File| menu libovolného okna
prostředí.

Všechna makra a funkce, které knihovna definuje pro přímé volání uživatelem jsou
\emph{exportována} z \emph{package} \verb|exil|. Před interakcí s knihovnou je
tedy třeba vstoupit do package \verb|exil-user|, který symboly z~package
\verb|exil| \emph{importuje}. Symboly z~package je také možno importovat do
existujícího package takto:
\begin{minted}{cl}
  (defpackage :my-package
    (:documentation "user-defined package")
    (:use :common-lisp :exil)
    (:shadowing-import-from :exil :assert :step))
\end{minted}
Package \verb|exil| exportuje několik symbolů, které již v package
\verb|common-lisp| existují. Ty je třeba \emph{zastínit}, jak je vidět z ukázky.

\clearpage
%%%%%%%%%%%%%%%%%%%%%%%%%%%%%%%%%%%%%%%%%%%%%%%%%%%%%%%%%%%%%%%%%%%%%%%%%%%%%%%%
\subsection{Uživatelská příručka}
\subsubsection{Základní pojmy}
\begin{framed}
  \begin{itemize}
    \item TODO: v teoretické části uvést pojmy znovu s citacemi
    \item TODO: production memory v exilu splývá s pravidly znalostní báze - na
      příhodném místě uvést, že v teorii a některých systémech se rozlišuje
  \end{itemize}
\end{framed}

Nyní stručně zadefinuji základní pojmy, nutné pro pochopení fungování knihovny
ExiL a práci s ní. Význam pojmů bude jasnější, jakmile si je ukážeme na
příkladech. K těmto pojmům se posléze vrátím i~v~teoretické části textu
a~jejich popis rozšířím o další souvislosti.

První dva pojmy staví na pojmu znalost, který chápeme intuitivně a nebudu se jej
ani snažit definovat, nikoli na následujícím pojmu znalosti, jak ji chápeme
v~ExiLu (v takovém případě by byla definice cyklická).

Pojem expertního systému zatím chápejme tak, jak jsem jej představil v úvodu
práce. V praktické části rozeberu pojem v potřebné šíři.
\begin{description}[leftmargin=6cm,style=sameline,align=right,labelsep=0.5cm]
  % \item[problémová doména] množina pojmů relevantních pro řešení určité skupiny
  %   problémů
  \item[fakt] elementární statická znalost - tvrzení
  \item[(odvozovací) pravidlo] elementární odvozovací znalost - pokud víme, že
    (ne)platí nějaká tvrzení, můžeme odvodit, že platí i~nějaká další
  \item[znalost (v ExiLu)] množina faktů a pravidel
  \item[znalostní báze] výchozí znalost expertního systému
  \item[pracovní paměť] aktuální množina faktů
  % \item[production memory] aktuální množina pravidel
  \item[inference] odvozování - postupná aplikace odvozovacích pravidel
\end{description}
Pojem \emph{pracovní paměť} není příliš intuitivní. Jde o doslovný překlad
v~literatuře užívaného pojmu \emph{working memory}, kterým je označována množina
faktů (tvrzení), které expertní systém v danou chvíli považuje za platné. Nejde
tedy ve skutečnosti o paměť, nýbrž o obsah pomyslné paměti. Pojem pracovní
množina faktů by byl jistě výstižnější, bohužel ale také značně těžkopádný.

%%%%%%%%%%%%%%%%%%%%%%%%%%%%%%%%%%%%%%%%%%%%%%%%%%%%%%%%%%%%%%%%%%%%%%%%%%%%%%%%
\subsubsection{Struktura programu}

\begin{listing}[h]
\caption{Základní struktura exilového programu}
\label{typical structure}
\begin{clcode}
;;; definition of knowledge base
;; facts
(deffacts world
  (in box A)
  (in robot B)
  (goal move box A B))

;; inference rules
(defrule move-robot
  (goal move ?object ?from ?to)
  (in ?object ?from)
  (- in robot ?from)
  (in robot ?z)
  =>
  (retract (in robot ?z))
  (assert (in robot ?from)))

(defrule move-object
  (goal move ?object ?from ?to)
  ?rob-pos <- (in robot ?from)
  ?obj-pos <- (in ?object ?from)
  =>
  (retract ?rob-pos)
  (retract ?obj-pos)
  (assert (in robot ?to))
  (assert (in ?object ?to)))

(defrule stop
  (goal move ?object ?from ?to)
  (in ?object ?to)
  =>
  (halt))

;;; initialization of working memory
(reset)

;;; inference execution
(run)
\end{clcode}
\end{listing}

Příklad \ref{typical structure} na straně \pageref{typical structure} ukazuje
minimální strukturu programu nad knihovnou ExiL (dále exilový program). První
část programu tvoří definice znalostní báze. Ta sestává z definic faktů, ze
kterých expertní systém vychází, a definic odvozovacích pravidel, jež jsou
následně aplikována při inferenci.

Definice faktů jsou uspořádány do skupin označených názvem (v tomto případě
\verb|world|). V ukázkovém programu si snadno vystačíme s jednou skupinou faktů,
v reálných programech bude ale těchto skupin většinou více. Tato organizace
umožňuje snadnou redefinici, případně odebrání, jen některých skupin faktů v
případě potřeby.  Definice skupiny faktů \verb|world| v příkladu přidává do
znalostní báze informaci o~počáteční pozici robota, krabice a~o~našem záměru
přesunout krabici z~pozice \verb|A| na pozici \verb|B|.

Následuje definice odvozovacích pravidel. Definice každého pravidla sestává
z~množiny podmínek, tedy předpokladů pro jeho splnění (a následnou aktivaci),
a~množiny důsledků, tedy libovolných lispových lispových výrazů, které jsou při
aktivaci pravidla vyhodnoceny.  Tyto dvě množiny jsou od sebe odděleny
\emph{symbolem}~\verb|=>|.

Podmínky odvozovacích pravidel jsou ve formě vzorů (\emph{pattern}). Struktura
vzorů je stejná jako struktura faktů (viz sekce \ref{knowledge base
definition}), ale na rozdíl od nich mohou obsahovat proměnné (symboly začínající
otazníkem).  Při vyhodnocování podmínek pravidla je zajišťěna
konzistence vazeb těchto proměnných a výskyty všech proměnných v~důsledcích
pravidla jsou při jeho aktivaci nahrazeny jejich vazbami. Detaily viz sekce
\ref{inference}

Důsledky pravidel typicky obsahují příkazy pro modifikaci pracovní paměti (viz
sekce \ref{modifikace}), tedy přidání (\verb|assert|), odebrání
(\verb|retract|), či úpravu (\verb|modify|) faktů v ní. Nemusí tomu tak ale být
vždycky - důsledkem aktivace pravidla může být např. vypsání výstupu, logování,
zápis souboru, ale také např. ovládání externího systému.

Ukázkový příklad definuje tři odvozovací pravidla. Pravidlo
\verb|move-robot| je aktivováno, pokud chceme přesunout nějaký objekt z pozice
\verb|?from| na pozici \verb|?to|, objekt se nachází v pozici \verb|?from|
a~robot nikoli (třetí podmínka je negovaná, viz sekce \ref{inference}).
Poslední podmínka slouží pouze k navázání původní pozice robota.  Při aktivaci
pravidla je v pracovní paměti nahrazena informace o~původní pozici robota
pozicí \verb|?from|. Robot se tedy nyní nachází na stejné pozici, jako kýžený
objekt.

Podmínky pravidla \verb|move-object| vyžadují, aby byl jak robot, tak objekt
určený k přesunu, na pozici \verb|?from|. Při jeho aktivaci je robot i s objektem
přesunut na pozici \verb|?to| nahrazením faktů o původních pozicích novými,
podobně jako v prvním pravidle. Definice pravidla obsahuje speciální notaci (s
použitím operátoru \verb|<-|), jejímž účelem je navázání celého faktu na
proměnnou. Ten pak můžeme v důsledcích snadno ostranit z pracovní paměti.
Detaily opět viz sekce \ref{inference}

Poslední pravidlo slouží k zastavení inference, pokud se již objekt nachází na
cílové pozici. Inference je zde zastavena explicitním voláním \verb|(halt)|.
Druhou možností by bylo odstranit z pracovní paměti fakt definující cíl, neboť v
takovou chvíli nemůže být žádné další pravidlo splňeno.

Jakmile je znalostní báze nadefinována, můžeme z ní inicializovat pracovní
paměť. To provedeme voláním \verb|(reset)|, které (po případném vyčištění
původních faktů) přidá do pracovní paměti fakty ve všech definovaných skupinách.

Poslední nutnou fází exilového programu je spuštění inference. To můžeme udělat
nejjednodušeji voláním \verb|(run)|. Inferenční mechanismus poté postupně
vyhodnocuje, která odvozovací pravidla mají splněné všechny podmínky, v každém
kroku z nich jedno vybere a aktivuje jej. Detaily viz sekce \ref{inference}

Výstup programu je následující:
\begin{minted}{cl}
==> (IN ROBOT B)
==> (IN BOX A)
==> (GOAL MOVE BOX A B)
Firing MOVE-ROBOT
<== (IN ROBOT B)
==> (IN ROBOT A)
Firing MOVE-OBJECT
<== (IN ROBOT A)
<== (IN BOX A)
==> (IN ROBOT B)
==> (IN BOX B)
Firing STOP
Halting
\end{minted}
Řádky začínající symbolem \verb|==>| označují fakty přibyvší do pracovní paměti,
řádky začínající \verb|<==| fakty z paměti odstraněné. Tento výstup obdržíme
pouze pokud zapneme sledování faktů voláním \verb|(watch facts)| (viz sekce
\ref{inference tracing}). První tři fakty přibydou do pracovní paměti při
vyhodnocení volání \verb|(reset)|, další pak spolu s postupnou aplikací
odvozovacích pravidel. Dotážeme-li se po skončení inference na seznam faktů v
pracovní paměti voláním \verb|(facts)|, obdržíme výstup
\cl|((GOAL MOVE BOX A B) (IN ROBOT B) (IN BOX B)).|
Robot i krabice jsou tedy na cílové pozici.

Kód exilového programu má deklarativní charakter. Nikde jsme nemuseli
specifikovat, jakou posloupností akcí má systém k výsledku dospět. To nás ovšem
nezbavuje nutnosti chápat fungování inferenčního mechanismu ExiLu. Nebudeme-li
při konstrukci programu opatrní, může výpočet snadno dospět k neočekávaným
výsledkům, dostat se do slepé větve, či se zacyklit. Tyto problémy jsou často
způsobeny nezamýšlenou interferencí podmínek pravidel s důsledky jiných.

\FloatBarrier

\subsubsection{Definice znalostní báze}
\label{knowledge base definition}

ExiL, stejně jako CLIPS, rozlišuje dva typy faktů - jednoduché (\emph{simple,
ordered}) a strukturované (\emph{templated}). Stuktura jednoduchého faktu je udána
pouze pořadím \emph{atomů}, typickou volbou je např. \verb|objekt-attribut-hodnota|:
\cl|(box color red),| či \verb|relace-objekty|: \cl|(in box hall).|

Strukturované fakty mají naproti tomu explicitně pojmenované složky (sloty).
Typicky popisují objekt s množinou pojmenovaných atributů: \cl|(box :color red :size small),|
či relaci s pojmenovanými aktory: \cl|(in :object box :location hall),| kde
\verb|box| a \verb|in| jsou šablony (\emph{template}), které je třeba definovat
předem. Na pořadí specifikace slotů u strukturovaných faktů
nezáleží.

Vyjadřovací síla obou typů faktů je stejná, použitím explicitnějších
strukturovaných faktů ale docílíme lepší čitelnosti a jednoznačnější sémantiky
exilového programu, zláště třeba v případě relací na jedné množině objektů:
\cl|(father john george).|

Šablonu definujeme voláním \emph{makra} \verb|deftemplate|, např:
\cl|(deftemplate in object (location :default here)).| Prvním parametrem je
název šablony, za ním následuje libovolný počet specifikací slotů. Specifikací
slotu je buď symbol - jméno slotu, nebo \emph{seznam}, jehož hlavou
(\emph{car}) je jméno slotu a~tělem (\emph{cdr}) je \emph{property list (plist)}
s dalšími parametry. Aktuálně systém umožňuje pouze specifikaci výchozí hodnoty
slotu \emph{klíčem} \verb|:default|. Ta je použita, není-li při specifikaci
faktu, používajícího tuto šablonu, uvedena hodnota pro daný slot.

Je-li už šablona požadovaného názvu definována, ale neexistují v pracovní
paměti fakty, které ji používají, je její stávající definice nahrazena. Pokud
ale v~pracovní paměti existují takové fakty, skončí volání \verb|deftemplate|
výjimkou.

Seznam názvů všech definovaných šablon můžeme získat voláním \verb|(templates)|.
Specifikaci šablony pak získáme voláním makra
\verb|find-template|, např. \verb|(find-template in).| Definici šablony zrušíme
voláním makra \verb|undeftemplate|, např. \verb|(undeftemplate goal)|. To opět
skončí výjimkou, existují-li v pracovní paměti fakty, které šablonu využívají.

Fakty, ze kterých expertní systém vychází, zavádíme pomocí skupin faktů. Ty
definujeme makrem \verb|deffacts|, např.:
\begin{minted}{cl}
(deffacts initial
  (goal move box A B)
  (in :object box :location A))
\end{minted}
Prvním parametrem je název skupiny, pak následuje libovolný počet specifikací
faktů. Opakovaným voláním makra \verb|deffacts| je skupina faktů
redefinována.

Specifikace faktu je vždy tvořena seznamem. Pokud jde o jednoduchý fakt,
specifikací je prostě seznam atomů. Jde-li o fakt strukturovaný, je prvním
prvkem specifikace název šablony, za ním následuje plist určující hodnoty slotů
faktu. Pokud není hodnota některého slotu uvedena, je buď použita
výchozí hodnota, pokud byla v šabloně specifikována, nebo hodnota \verb|nil| v
opačném případě.

Seznam názvů všech definovaných skupin faktů získáme voláním
\verb|(fact-groups)|. Specifikaci skupiny pak voláním makra
\verb|find-fact-group|, např. \verb|(find-fact-group initial)|. Ke zrušení
definice skupiny slouží makro \verb|undeffacts| (voláme s názvem skupiny).

Pravidla, pomocí nichž expertní systém během inference odvozuje nové fakty,
definujeme makrem
\verb|defrule|, např.:
\begin{minted}{cl}
(defrule move-robot
  (goal :action move :object ?obj :from ?from)
  (in :object ?obj :location ?from)
  (- in :object robot :location ?from)
  ?robot <- (in :object robot :location ?)
  =>
  (modify ?robot :location ?from)).
\end{minted}
Podmínková část pravidla (před symbolem \verb|=>|) je tvořena vzory. Ty mohou
být, stejně jako fakty, jednoduché, nebo strukturované. Kromě toho umožňuje
definice pravidla několik speciálních konstruktů (negace podmínky, navázání
proměnné na celou podmínku). Ty popíšu podrobně v sekci \ref{inference} spolu~s
tím, jak jsou podmínky pravidla při inferenci vyhodnocovány.

Důsledkovou část pravidla (za symbolem \verb|=>|) tvoří libovolný počet
lispových výrazů. Jak se tyto vyhodnocují popíšu opět v sekci \ref{inference}.

Opakovaným voláním makra \verb|defrule| odvozovací pravidlo redefinujeme.
K~získání seznamu názvů definovaných pravidel a jejich specifikací slouží
\emph{funkce} \verb|rules| a makro \verb|find-rule|, podobně jako u šablon a
skupin faktů.  Ke zrušení definice pravidla slouží makro \verb|undefrule|.

\subsubsection{Modifikace pracovní paměti}
\label{modifikace}

Pracovní paměť je množina faktů, které systém v danou chvíli považuje za platné.
Její obsah můžeme vypsat voláním funkce \verb|facts|. Funkcí \verb|reset|
inicializujeme pracovní paměť ze znalostní báze. Jejím voláním jsou do pracovní
paměti zavedeny fakty všech definovaných skupin faktů (viz sekce
\ref{knowledge base definition}).

Obsah pracovní paměťi může být dále modifikován třemi makry:
\begin{itemize}
  \item \verb|assert| přidává fakt(y) do pracovní paměti,
  \item \verb|retract| fakt(y) z pracovní paměti odebírá a
  \item \verb|modify| přímo modifikuje existující fakty.
\end{itemize}
Ta lze volat buď před započetím inference (ale po volání \verb|reset|, neboť to
dodatečné úpravy vymaže), nebo v jejím průběhu, pokud inferenci krokujeme (viz
sekce \ref{inference}). Makra také typicky voláme v důsledcích pravidel.

Makra \verb|assert| a \verb|retract| berou jako parametry libovolný počet
specifikací faktů ve stejném formátu, jako u makra \verb|deffacts| (ale bez
názvu skupiny). Makro \verb|modify| lze použít jen u strukturovaných faktů. Toto
makro bere jako první parametr specifikaci faktu, zbytek parametrů tvoří plist
určující hodnoty slotů ke změně. Např.
\begin{minted}{cl}
(modify (in :object box :location A) :location B)
\end{minted}
nahradí v pracovní paměti fakt \verb|(in :object box :location A)| faktem
\verb|(in :object box :location B)|.
Toto makro je obzvláště užitečné, navážeme-li v podmínkách pravidla celý fakt na
proměnnou (viz sekce \ref{inference}).

Všechny fakty můžeme z pracovní paměti odebrat voláním funkce \verb|retract-all|
(bez parametrů). To je ale zřídka užitečné, typicky použijeme spíše funkci
\verb|reset| pro navrácení pracovní paměti do výchozího stavu.

\subsubsection{Inference}
\label{inference}

Inference (odvozování nových faktů z aktuálních) probíhá v krocích. V každém
kroku jsou vyhodnoceny podmínky všech pravidel, načež je ze splněných pravidel
vybráno jedno, které je posléze aktivováno. Inferenční kroky můžeme buď spouštět
jednotlivě voláním \verb|(step)|, nebo voláním \verb|(run)| spustit
cyklus, který provádí inferenční kroky, dokud je to možné. Cyklus je buď
přerušen ve chvíli, kdy není splněno žádné další pravidlo, nebo voláním
\verb|(halt)| v důsledcích právě aktivovaného pravidla.

Podmínky pravidel jsou ve tvaru vzorů a jsou spojeny logickou konjunkcí,
pravidlo je tedy splněno, jsou-li splněny všechny jeho podmínky. Kromě
toho mohou být některé podmínky negovány. Taková podmínka je splněna tehdy,
neexistuje-li v pracovní paměti žádný fakt, který by se shodoval s jejím vzorem
(při zachování konzistence vazeb proměnných). Negovanou podmínku značí znak
\verb|-| (minus) na prvním místě specifikace vzoru.

Vyhodnocování podmínek pravidel probíhá ve dvou fázích. V první fází srovnáváme
vzory jednotlivých podmínek se všemi fakty v pracovní paměti a~to pouze
strukturálně, tedy bez ohledu na vazby proměnných. Prvním požadavkem shody je u
jednoduchých faktů stejná délka (počet atomů), u strukturovaných faktů stejná
šablona. Jednoduchý fakt se nikdy nemůže shodovat se strukturovaným vzorem a
naopak.

Dále jsou pak porovnávány jednotlivé atomy (u jednoduchých) či
sloty (u~složených) faktu vůči odpovídajícímu atomu (slotu) vzoru. Není-li atom
(slot) vzoru proměnná, je jednoduše porovnán s atomem faktu. Je-li atomem
proměnná, považujeme jej v této fázi automaticky za shodu. Například vzor
\cl|(in :object robot :location ?loc)|
se shoduje s faktem
\cl|(in :object robot :location A)|
nikoli však s fakty
\begin{minted}{cl}
(in :object box :location A)
(is-in :object robot :location A)
(in robot A).
\end{minted}
Tímto předvýběrem
tedy získáme ke každé podmínce pravidla množinu faktů, které mají stejnou
strukturu a stejné hodnoty neproměnných atomů.

Ve druhé fázi vyhodnocování hledáme z předvybraných faktů takovou posloupnost
(délka odpovídá počtu podmínek pravidla), kde po spárování s odpovídajícími
vzory podmínek obdržíme konzistentní vazby proměnných. To znamená, že
vyskytuje-li se v podmínkách pravila některá proměnná vícekrát, musí mít
odpovídající fakty na daných pozicích stejný atom.  Mějme například pravidlo
s~podmínkami
\begin{minted}{cl}
(goal move ?obj ?from ?to)
(in :object ?obj  :location ?from)
(in :object robot :location ?to)
\end{minted}
Vzor první podmínky je jednoduchý, zatímco další dva jsou strukturované. To ale
ničemu nevadí, je třeba pouze najít fakty odpovídající struktury. Posloupnost
faktů
\begin{minted}{cl}
(goal move box A B)
(in :object box   :location B)
(in :object robot :location A)
\end{minted}
neprojde druhou fází výběru, neboť vazby proměnných nejsou konzistentní.
Proměnná \verb|?from| je například v první podmínce navázána na symbol \verb|A|, v
druhé ale na \verb|B|. Kdyby si ovšem krabice s robotem vyměnily pozice, budou
vazby proměnných konzistentní a podmínky pravidla budou splněny. Proměnná
\verb|?from| by pak nabyla hodnoty \verb|A|, proměnná \verb|?to| hodnoty
\verb|B| a proměnná \verb|?obj| hodnoty \verb|box|.

Vyhodnocení negovaných podmínek si můžeme představit tak, že nejprve vyhodnotíme
a navážeme proměnné všech ostatních podmínek. Pokud poté neexistuje fakt, který
by se se vzorem negované podmínky shodoval a měl konzistentní vazby se zbytkem navázaných
proměnných, je tato podmínka splněna. Mějme například pravidlo s~podmínkami
\begin{minted}{cl}
(goal move box ?from ?to)
(in box ?from)
(- in robot ?from).
\end{minted}
Máme-li v pracovní paměti pouze fakty
\begin{minted}{cl}
(goal move box A B)
(in box A)
(in robot B),
\end{minted}
budou podmínky pravidla splněny, neboť po spárování vzorů prvních dvou podmínek
s prvními dvěma fakty bude proměnná \verb|?from| navázána na hodnotu \verb|A|
a~neexistuje fakt, který by se shodoval se vzorem \verb|(in robot A)|. Přesuneme-li
ale robota na pozici \verb|A|, podmínka již splněna nebude a pravidlo nelze
aktivovat.

Ve vzorech podmínek pravidla můžeme využít speciální proměnné \verb|?|.
Konzistence vazby této proměnné není při vyhodnocování testována, takže
vyskytuje-li se tato proměnná na více místech, chová se tak, jako kdyby byl
každý výskyt označen unikátním názvem (podobně jako proměnná \verb|_| v
Prologu). Použitím této proměnné dáváme najevo, že nás konkrétní hodnota daného
atomu nazajímá. Ve strukturovaných vzorech podmínek není třeba tyto sloty uvádět,
neboť \verb|?| je výchozí hodnotou slotu vzoru.

Posledním speciálním konstruktem je navázání celého faktu na proměnnou.
Například pravidlo
\begin{minted}{cl}
(defrule move
  ?fact <- (in :object ? :location A)
  =>
  (modify ?fact :location B))
\end{minted}
přesune každý objekt z pozice \verb|A| na pozici \verb|B|. Na proměnnou
můžeme navázat i~jednoduchý fakt, pak ale nemůžeme použít makra \verb|modify|.
Můžeme ovšem volat \verb|(retract ?fact)|, neboť proměnná \verb|?fact| je při
aktivaci pravidla nahrazena specifikací faktu, který byl se vzorem podmínky
spárován.

Podmínky některých pravidel mohou být při vyhodnocování splněny několika různými
posloupnostmi faktů. Výsledkem vyhodnocování pravidel tedy není pouze množina
splněných pravidel, nýbrž množina shod (\emph{match}). Každá shoda je tvořena
pravidlem a \emph{subsitucí} proměnných navázaných při vyhodnocování jeho
podmínek.  Subsituci chápejme jako zobrazení z množiny všech proměnných,
vyskytujících se v podmínkách pravidla, na konkrétní hodnoty atomů (slotů).
Aplikací této substituce na vzory podmínek pravidla získáme opět posloupnost
faktů, kterými byly podmínky v dané shodě splněny. Tato substituce je následně
použita při aktivaci pravidla k náhradě proměnných v jeho důsledích.

ExiL uchovává aktuální množinu shod. Ta ve skutečnosti není přepočítána v~první
fázi inferenčního kroku, jak jsem dosud pro jednoduchost tvrdil, nýbrž
automaticky po každé změně pracovní paměti či množiny pravidel (detaily viz
RETE).  Aktuální množinu shod nazývám, po vzoru CLIPSu, \emph{agenda} a lze ji
vypsat voláním stejnojmenné funkce. Každá shoda v agendě je opatřena časovým
razítkem, shody je tedy možné uspořádat podle toho, kdy do agendy přibyly.

Je-li na začátku inferenčního kroku v agendě více shod, je třeba z nich jednu
vybrat k aktivaci. Výběr shody záleží na zvolené strategii. ExiL poskytuje
následující strategie výběru shody:
\begin{description}[leftmargin=5cm,style=sameline,align=right,labelsep=0.5cm]
  \item[depth-strategy] vybírá shodu, která do agendy přibyla nejpozději
  \item[breadth-strategy] vybírá shodu, která do agendy přibyla jako první
  \item[simplicity-strategy] vybere shodu, jejíž pravidlo má nejméně podmínek
  \item[complexity-strategy] volí shodu, jejíž pravidlo má nejvíce podmínek.
\end{description}
Názvy prvních dvou strategií vychází z toho, že jde o prohledávání prostoru
stavů systému do hloubky, či do šířky, jak blíže popíšu v teoretické části
textu, spolu s~motivacemi pro využití jednotlivých typů strategií a dalšími typy
strategií, které se v expertních systémech používají.

Výchozí strategií je \verb|depth-strategy|. Strategii, která bude v inferenci
použita, můžeme zvolit voláním makra \verb|setstrategy| s názvem strategie,
např. \verb|(setstrategy breadth-strategy)|. Seznam názvů strategií můžeme
vypsat voláním \verb|(strategies)|, název aktuálně zvolené strategie pak voláním
\verb|(current-strategy)|.

\begin{framed}
  \begin{itemize}
    \item fáze podrobně
    \begin{itemize}
      \item výběr pravidla - strategie
      \item aktivace - vyhodnocení důsledků (typicky modifikace pracovní paměti)
        navázání proměnných, eval
    \end{itemize}
    \item spuštění inference, krokování (může se prolínat s ručnímodifikací w.m.)
    \item volání (step) a (run), když už nelze dále krokovat
    \item queries - agenda, strategies
  \end{itemize}
\end{framed}


%%%%%%%%%%%%%%%%%%%%%%%%%%%%%%%%%%%%%%%%%%%%%%%%%%%%%%%%%%%%%%%%%%%%%%%%%%%%%%%%
\subsubsection{Reset prostředí}
\begin{framed}
  \begin{itemize}
    \item durable/volatile slots
    \item clean, reset, complete reset (neměl by se jmenovat complete clean?)
  \end{itemize}
\end{framed}

%%%%%%%%%%%%%%%%%%%%%%%%%%%%%%%%%%%%%%%%%%%%%%%%%%%%%%%%%%%%%%%%%%%%%%%%%%%%%%%%
\subsubsection{Sledování průběhu inference}
\label{inference tracing}
\begin{framed}
  \begin{itemize}
    \item watchery
  \end{itemize}
\end{framed}

%%%%%%%%%%%%%%%%%%%%%%%%%%%%%%%%%%%%%%%%%%%%%%%%%%%%%%%%%%%%%%%%%%%%%%%%%%%%%%%%
\subsubsection{Undo/redo}
\begin{framed}
  \begin{itemize}
    \item lze použít na všechny funkce/makra s vedlejším efektem
    \item pokud funkce nemá vedlejší efekt (fakt neexistuje, apod.), nezapíše se
      undo step
    \item queries - undo-stack, redo-stack
  \end{itemize}
\end{framed}

%%%%%%%%%%%%%%%%%%%%%%%%%%%%%%%%%%%%%%%%%%%%%%%%%%%%%%%%%%%%%%%%%%%%%%%%%%%%%%%%
\subsubsection{Zpětné řetězení}
\begin{framed}
  \begin{itemize}
    \item cíle jako patterny
    \item základní inference - nejdřív fakty, pak pravidla, v jakém pořadí
      vybírá
    \item alternativní odpovědi - backtracking
  \end{itemize}
\end{framed}

%%%%%%%%%%%%%%%%%%%%%%%%%%%%%%%%%%%%%%%%%%%%%%%%%%%%%%%%%%%%%%%%%%%%%%%%%%%%%%%%
\subsubsection{Práce s více prostředími}
\label{multiple environments}

%%%%%%%%%%%%%%%%%%%%%%%%%%%%%%%%%%%%%%%%%%%%%%%%%%%%%%%%%%%%%%%%%%%%%%%%%%%%%%%%
\subsubsection{Grafické uživatelské rozhraní}

% \clearpage
% \subsection{Referenční příručka}

V uživatelské příručce jsem uvedl všechny možnosti, které knihovna ExiL
poskytuje. Zde tedy uvedu pouze seznam signatur funkci a maker s odkazy na
opovídající sekce příručky, pro snadnou referenci.

\vspace{0.5cm}

\textbf{Definice šablon} - \ref{knowledge base definition}, \ref{guide clips}
\begin{minted}{cl}
(defmacro deftemplate (name &body slots))
(defun deftemplatef (name slots))
(defmacro undeftemplate (name))
(defun undeftemplatef (name))
(defun templates ())
(defmacro find-template (name))
(defun find-templatef (name))
\end{minted}

\textbf{Definice skupin faktů} - \ref{knowledge base definition}, \ref{guide clips}
\begin{minted}{cl}
(defmacro deffacts (name &body fact-specs))
(defun deffactsf (name fact-specs))
(defmacro undeffacts (name))
(defun undeffactsf (name))
(defun fact-groups ())
(defmacro find-fact-group (name))
(defun find-fact-groupf (name))
\end{minted}

\textbf{Definice pravidel} - \ref{knowledge base definition}, \ref{guide clips}
\begin{minted}{cl}
(defmacro defrule (name &body body))
(defun defrulef (name body))
(defmacro undefrule (name))
(defun undefrulef (name))
(defun rules ())
(defmacro find-rule (name))
(defun find-rulef (name))
(defun agenda ())
\end{minted}

\textbf{Modifikace pracovní paměti} - \ref{modifikace}
\begin{minted}{cl}
(defun facts (&optional start-index end-index at-most))
(defmacro assert (&rest fact-specs))
(defun assertf (&rest fact-specs))
(defmacro retract (&rest fact-specs))
(defun retractf (&rest fact-specs))
(defun retract-all ())
(defmacro modify (fact-spec &rest mod-list))
(defun modifyf (fact-spec mod-list))
\end{minted}

\textbf{Sledování průběhu inference} - \ref{inference tracing}
\begin{minted}{cl}
(defmacro watch (watcher))
(defun watchf (watcher))
(defmacro unwatch (watcher))
(defun unwatchf (watcher))
(defmacro watchedp (watcher))
(defun watchedpf (watcher))
\end{minted}

\textbf{Definice cílů} - \ref{backward inference}
\begin{minted}{cl}
(defmacro defgoal (goal-spec))
(defun defgoalf (goal-spec))
(defmacro undefgoal (goal-spec))
(defun undefgoalf (goal-spec))
(defun goals ())
\end{minted}

\textbf{Spuštění inference} - \ref{inference}, \ref{backward inference}
\begin{minted}{cl}
(defun step ())
(defun halt ())
(defun run ())
(defun back-step ())
(defun back-run ())
\end{minted}

\textbf{Reset prostředí} - \ref{env cleanup}
\begin{minted}{cl}
(defun clear ())
(defun reset ())
(defun complete-reset ())
\end{minted}

\textbf{Undo/redo} - \ref{undo}
\begin{minted}{cl}
(defun undo ())
(defun redo ())
(defun undo-stack ())
(defun redo-stack ())
\end{minted}

\textbf{Práce s více prostředími} - \ref{multiple environments}
\begin{minted}{cl}
(defmacro defenv (name &key redefine))
(defun defenvf (name &key redefine))
(defmacro undefenv (name))
(defun undefenvf (name))
(defmacro setenv (name))
(defun setenvf (name))
(defun environments ())
(defun current-environment ())
\end{minted}

\textbf{Inferenční strategie} - \ref{inference}
\begin{minted}{cl}
(defmacro setstrategy (name))
(defun setstrategyf (name))
(defun current-strategy ())
(defun strategies ())
\end{minted}

\textbf{Grafické uživatelské rozhraní} - \ref{guide gui}
\begin{minted}{cl}
(defun show-gui (&optional environment))
\end{minted}

\clearpage
\subsection{Implementace}
\begin{framed}
Implementace or v podmínkách pravidla by vyžadovala výraznou změnu v algoritmu,
který vytváří síť RETE, neboť join node je problematické znovu využít.
Implementace forall by vyžadovala změnu vyhodnocování vazeb proměnných a celkově
join algoritmu. Zvážit problémy implementace všech podobných rozšíření - or,
and, negace celé konjunkce či disjunkce, exists, forall, viz
\url{http://www.csie.ntu.edu.tw/~sylee/courses/clips/bpg/node5.4.html}
\end{framed}

\begin{framed}
  \begin{itemize}
    \item architektura programu (objektový návrh, nedostatky Lispu, je
      environment jako god class důsledkem těchto problémů?)
    \item síť RETE, její vytváření $\rightarrow$ výhody - sdílení uzlů
    \item implementovaná rozšíření
    \begin{itemize}
      \item undo/redo - kopie sítě rete, testování ekvivalence
      \item zpětné řetězení - zásobníky, backtracking, obchází rete
      \item kompozitní podmínky (not, and, or, forall) - neimplementováno -
        popsat, jak výrazné změny rete by vyžadovalo a jak by se projevilo na
        efektivitě (hlavní výhodou rete je sdílení uzlů, které by zde bylo dost
        problematické)
      \item syntaktický mód - parser, implementace složených podmínek pro atom
        (\~{}asdf\&asf) by vyžadovalá podobné změny jako implementace kompozitních
        podmínek
      \item gui - capi, propojení s environmentem - notify
    \end{itemize}
  \end{itemize}
\end{framed}


\clearpage
\section{Teoretická část}
\begin{framed}
  \begin{itemize}
    \item rozebrat, jak řeší řeší GPS v PoAIP problémy se zpětným řetězením
    \item MYCIN, jakožo ES se zpětným řetězením, neřeší negativní znalost vůbec
    \item co se týče kompozitních podmínek, používá MYCIN and-or stromy, šlo by
      aplikovat v rete?
    \item CLIPS basic programming guide - defrule construct - conflict resolution
      strategies, LHS conditional elements
  \end{itemize}
\end{framed}
\input{expertni-system}

\clearpage
\bibliographystyle{czechiso}
% \bibliographystyle{plainnat}
\bibliography{reference}
\end{document}
