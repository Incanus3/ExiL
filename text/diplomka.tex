\documentclass[12pt]{article}
% tables - list of tables, joinlists - join list of tables and figures
\usepackage[master,figures]{updiplom}
\usepackage[utf8]{inputenc}
% \usepackage{czech}
% \usepackage{epsfig}
\usepackage{url}
\usepackage{hyperref}
\hypersetup{backref,
linktocpage,
colorlinks=true,
linkcolor=blue,
citecolor=blue,
urlcolor=blue}

% \usepackage{listings}
\usepackage{minted}
\usemintedstyle{autumn}
% \usepackage{natbib}
% \DeclareUrlCommand\url{\def\UrlLeft{<}\def\UrlRight{>} \urlstyle{tt}}

\newcommand{\registered}{\textsuperscript{\textregistered}}

\title{Implementace expertního systému v jazyce Common Lisp}
\author{Jakub Kaláb}
\year{2013}
%\date{\today}
\date{31.12.2013}
\annotation{
Expertní systémy mají v~praxi bohaté využití. Jejich smyslem je
asistovat expertovi na danou problematiku, či jej plně nahradit. V~příloze
bakalářské práce implementuji prázdný expertní systém s~dopředným řetězením
inspirovaný systémem CLIPS jako knihovnu v~programovacím jazyku Common
Lisp tak, aby jej bylo možno plně integrovat do dalších programů.
}
\thanks{Děkuji Mgr. Martinu Dostálovi, Ph.D. za vedení této diplomové práce.}

%%%%%%%%%%%%%%%%%%%%%%%%%%%%%%%%%%%%%%%%%%%%%%%%%%%%%%%%%%%%%%%%%%%%%%%%%%%%%%%%
\begin{document}

\maketitle
\renewcommand\listoflistingscaption{Seznam ukázek kódu}
\renewcommand\listingscaption{Ukázka kódu}
\listoflistings
\newminted{cl}{linenos,frame=lines}

\nocite{introduction}
\nocite{paradigms}
\nocite{doorenbos}
\nocite{practical}
\nocite{clips}
\nocite{clhs}
\nocite{expert-system}
\nocite{rete}

\clearpage
%%%%%%%%%%%%%%%%%%%%%%%%%%%%%%%%%%%%%%%%%%%%%%%%%%%%%%%%%%%%%%%%%%%%%%%%%%%%%%%%
\section{Úvod}

Pojem expertního systému spadá do oblasti umělé inteligence. Jde o počítačový
systém, který simuluje rozhodování experta nad zvolenou problémovou doménou.
Expertní systém může experta zcela nahradit, nebo mu při rozhodování asistovat.

Ve své bakalářské práci \cite{bakalarka} jsem implementoval základní knihovnu pro tvorbu
expertních systémů (tzv. prázdný expertní systém) s~dopředným řetězením v~jazyce
Common Lisp. Cílem této práce je knihovnu rozšířit o následující:
\begin{itemize}
  \item syntaktický režim pro zajištění přiměřené kompatibility se systémem
    CLIPS,
  \item možnost vrácení provedených změn včetně odvozovacích kroků,
  \item podpora pro ladění s jednoduchým grafickým uživatelským rozhraním pro
    prostředí LispWorks\texttrademark,
  \item rozšíření odvozovacího aparátu o základní zpětné řetězení.
\end{itemize}

Jazyk Common Lisp\footnote{\url{http://en.wikipedia.org/wiki/Common\_Lisp}}
(případně jiné dialekty Lispu) je častou volbou pro implementaci umělé
inteligence díky svým schopnostem v oblasti symbolických výpočtů (manipulace
symbolických výrazů), na nichž řešení těchto problémů často staví. Navíc jde
o~velmi vysokoúrovňový, dynamicky typovaný jazyk, díky čemuž je programový kód
expresivní, snadno pochopitelný a tudíž jednoduše rozšiřitelný.

Synax systému CLIPS\footnote{\url{http://clipsrules.sourceforge.net}} byla
zvolena proto, že jde o reálně používaný
systém\footnote{\url{http://clipsrules.sourceforge.net/FAQ.html\#Q6}}, jehož
syntax je Lispu velmi blízká, takže není těžké ji v~Lispu napodobit.

Přestože běžnou praxí je začínat diplomovou práci teoretickou částí, definovat
jednotlivé pojmy a principy a ty poté v praktické části uplatnit, rozhodl jsem
se postupovat opačně, tedy začít práci praktickou částí. Domnívám se totiž (také
na základě zkušeností nabytých při vypracování bakalářské práce), že je podstatně
snazší, minimálně v řešené problematice, pochopit příklady bez detailní znalosti
teorie, než snažit se pochopit teorii bez příkladů, na nichž si lze popisované
pojmy a principy představit. V praktické části tedy uvedu jen minimální množtví
teorie nutné pro pochopení práce s knihovnou, načež se k ní v teoretické části
textu vrátím, pojmy zadefinuji přesně a rozšířím o souvislosti. V tuto chvíli už
si bude čtenář schopen představit, jaké problémy lze pomocí expertního systému
řešit a jak takový expertní systém v praxi vypadá.

\begin{framed}
  Pro popsaní některé teorie (symbolické výpočty, ...) je navíc užitečná (ne-li
  potřebná) představa, jak expertní systém funguje uvnitř.
\end{framed}

\begin{framed}
  Kdo už to dělal, čím se tohle liší
\end{framed}


\clearpage
\section{Praktická část}
Tato sekce popisuje knihovnu ExiL\footnotemark, která je výsledkem této
diplomové práce. Nejprve v uživatelské příručce popíšu její instalaci,
základní možnosti a typickou strukturu programu, který ji využívá. Poté
v~referenční příručce projdu všechny možnosti, které knihovna poskytuje.
Načež v části věnované implementaci popíšu architekturu jejího zdrojového
kódu a zmíním zajímavé části kódu implementující jednotlivá rozšíření.
Nakonec uvedu několik větších příkladů použití knihovny a~rozeberu několik
dalších možných rozšíření a co by obnášela z pohledu implementace.
\footnotetext{TODO: původ názvu}
%%%%%%%%%%%%%%%%%%%%%%%%%%%%%%%%%%%%%%%%%%%%%%%%%%%%%%%%%%%%%%%%%%%%%%%%%%%%%%%%
\subsection{Uživatelská příručka}
\subsubsection{Instalace}
- instalace - slime, sbcl, quicklisp, asdf, lispworks, získání kódu, git

Zdrojový kód knihovny je přiložen k~této diplomové práci a~lze jej také získat
zklonováním gitového\footnotemark{} repozitáře na adrese
\verb|git@github.com:Incanus3/ExiL.git| (v *nixových systémech např. zadáním
příkazu \verb|git clone git@github.com:Incanus3/ExiL.git|).
\footnotetext{http://git-scm.com/}

\subsubsection{Common Lisp}
- úvod do lispu - odkaz na practical common lisp, clhs

\subsubsection{Struktura programu}
popsat jednotlivé sekce kódu, jejich význam (korespondence s fázemi návrhu ES,
formulace problému, formát dat, vstupní znalosti, odvozovací krok, řízení
odvozování, ladění)
\begin{itemize}
  \item definice prostředí
  \item definice šablon - formát dat
  \item definice znalostní báze - vstupní znalost - deffacts, defrules
  \item (nastavení sledování průběhu inference - watchers)
  \item (úprava průběhu inference - strategie)
  \item spuštění / krokování inference - reset, run, step
  \item dotazy nad working memory - facts, agenda
  \item úprava working memory - assert, retract, modify
  \item dotazy nad znalostní bází - fact-groups, rules
  \item cleanup - volatile vs durable sloty prostředí
  \item undo/redo
  \item zpětné řetězení
  \item GUI
\end{itemize}

\begin{listing}[H] % position specifier as for figures
\caption{ExiL code example}
\label{example}
\begin{clcode}
(deftemplate goal action object from to)
(deftemplate in object location)

(deffacts world
  (in :object robot :location A)
  (in :object box :location B)
  (goal :action push :object box :from B :to A))

(defrule move
  (goal :action push :object ?obj :from ?from)
  (in :object ?obj :location ?from)
  (- in :object robot :location ?from)
  ?robot <- (in :object robot :location ?)
  =>
  (modify ?robot :location ?from))

(defrule push
  (goal :action push :object ?obj :from ?from :to ?to)
  ?object <- (in :object ?obj :location ?from)
  ?robot <- (in :object robot :location ?from)
  =>
  (modify ?robot :location ?to)
  (modify ?object :location ?to))

(defrule stop
  ?goal <- (goal :action push :object ?obj :to ?to)
  (in :object ?obj :location ?to)
  =>
  (retract ?goal)
  (halt))

(reset)

; (step)
(run)
\end{clcode}
\end{listing}


\clearpage
\section{Teoretická část}
\input{expertni-system}

\clearpage
\bibliographystyle{czechiso}
% \bibliographystyle{plainnat}
\bibliography{diplomka}
\end{document}
