\documentclass[a4paper,12pt]{article}
\usepackage[utf8x]{inputenc}
\usepackage{czech}
%\usepackage[czech]{babel}
\usepackage[]{updiplom}

%opening
\title{Implementace expertního systému s~dopředným řetězením}
\subtitle{Forward-chaining Expert System Implementation}
\author{Jakub Kaláb}
\year{2010}
\date{\today}
\annotation{Expertní systémy mají v~praxi bohaté využití. Jejich smyslem je
asistovat expertovi na danou problematiku, či jej plně nahradit. V~příloze
bakalářské práce implementuji prázdný expertní systém s~dopředným řetězením
inspirovaný systémem CLIPS jako knihovnu v~programovacím jazyku \emph{Common
Lisp} tak, aby jej bylo možno plně integrovat do dalších programů.}
\thanks{Děkuji Mgr. Martinu Dostálovi, PhD. za vedení této bakalářské práce.}

%%%%%%%%%%%%%%%%%%%%%%%%%%%%%%%%%%%%%%%%%%%%%%%%%%%%%%%%%%%%%%%%%%%%%%%%%%%%%%%%
\begin{document}

\maketitle

\section{Co je to experní systém}
Expertní systém je počítačový program, který ze zadané znalostní báze pomocí
zadaných pravidel odvozuje nová fakta. Díky tomu je jej možno využít v~praxi
jako asistenta odborníka v~daném oboru, nebo jej, v~ideálním případě, zcela
nahradit.

Pro lepší představu uvedu příklad - expertní systém v~ordinaci
praktického lékaře. Na začátku zadá lékař, případně za pomoci znalostního
inženýra, expertnímu systému znalostní bazi, tj. informace o~příznacích
známých chorob, možnostech jejich léčby, medikaci, konfliktních léků, atd.
Expertní systém potom při zadání příznaků pacienta s~jistou pravděpodobností
(odvislé od pravděpodobností zadaných v~jednotlivých pravidlech příznak
$\rightarrow$ choroba) určí možné choroby, alternativy léčby, apod.

\subsection{Co odlišuje expertní systém od jiných výpočetních programů}
Toto samozrejmě může dělat i~program, který bychom za expertní systém
neoznačili, rozdíl je v~tom, že expertní systém je obecný - znalostní
baze je zcela oddělena řídícího mechanismu. Tento je tudíž zcela
nezávislý na konkrétní doméně. Takto tomu u~jiných programů většinou
nebývá, data bývají smýchána s~rozhodovacím a~řídícím kódem programu
a~program je tudíž jednoúčelový.

\subsection{Charakteristické vlastnosti expertních systémů}

\newpage
%%%%%%%%%%%%%%%%%%%%%%%%%%%%%%%%%%%%%%%%%%%%%%%%%%%%%%%%%%%%%%%%%%%%%%%%%%%%%%%%

\section{Uživatelská příručka}
Tato práce se zabývá implementací již známých algoritmů, je tedy spíše
praktického zaměření. Úmyslně jsem tedy posunul kapitolu s~hlubším popisem
implementovaných algoritmů až za uživatelskou příručku, aby byl čternář při
její četbě již seznámen s~používanými datovými strukturami, strukturou
pravidel, apod.

Tato kapitola vyžaduje alespoň zběžnou znalost programovacího jazyka
Common Lisp, jeho základní datové typy (symbol, seznam) a metody pro
práci s~nimi.

V~kapitole budu občas uvádět v~závorkách a kurzívou názvy některých použitých
tříd či metod. Je to kvůli snadnější orientaci při následné četbě programátorské
dokumentace. Čtenář, který tuto číst nehodlá je může směle přeskakovat.

\subsection{Použité datové struktury}
Uživatel expertního systému se nejčasteji setká se dvěma datovými strukturami,
jsou to fakta (třída \emph{fact}) a pravidla (třída \emph{rule}). Fakta znalostní
baze mohou být buď jednoduchá (třída \emph{simple-fact}), specifikovaná prostým
seznamem atomů (např. \verb|(on book table)|), či strukturovaná (třída
\emph{template-fact}. Pro použití strukturovaných fakt je třeba nejdříve definovat
šablonu makrem \verb|deftemplate| následovně:
\begin{verbatim}
(deftemplate in (object location (ammount :default 1)))
\end{verbatim}
Specifikace takovéhoto strukturovaného faktu pak vypadá takto:
\verb|(in :object fridge :location kitchen)|.

Kromě fakt se uživatel setká ještě s~tzv. \emph{patterny} (pro neznalost
výstižného českého ekvivalentu se budu držet dobře zažitého výrazu anglického).
Tyto jsou faktům velmi podobné, jen se v~nich mohou vyskytovat proměnné.
Název proměnné v~mé implementaci vždy začíná otazníkem (stejně jako v~systému
CLIPS) a je tedy od běžného atomu jasně patrná. Patterny mohou být, stejně
jako fakta jednoduché či strukturované. Při snaze použít proměnnou v~popisu
faktu skončí vyhodnocení výrazu chybou.

Převodní pravidlo expertní systému se skládá ze dvou částí - podmínkek
a aktivací. Podmínky určují, za jakých okolností je pravidlo splněno
fakty znalostní baze a je jej možno zařadit do seznamu pravidel k~aktivaci
(tento seznam budu dále označovat jako \emph{agenda}). Seznam aktivací
je tvořen libovolným počtem Lispových výrazů (jež pochopitelně mohou
zahrnovat i systémem definované metody a makra), které se při aktivaci
pravidla vyhodnotí. Pravidlo tedy typicky při platnosti nějakých fakt
či neplatnosti jiných přidá do znalostní baze nějaká další fakta, či
některerá z~baze odebere.

\subsection{Manipulace s~fakty znalostní baze}
K~přidání faktu do znalostní baze souží makro \verb|assert|, přidání tedy
vypadá následovně:
\begin{verbatim}
(assert (on book table))
(assert (in :object fridge :location kitchen))
\end{verbatim}
Při pokusu o~opětované přidání již existujícího faktu se nestane nic.

Odebrání faktu z~baze se provádí makrem \verb|retract| se stejnou syntaxí.
Při pokusu o~odebrání neexistujícího faktu se opět nic nestane.

Posledním makrem z~této skupiny je \verb|modify| to slouží k~modifikaci
faktu (jeho nahrazení jiným) a používá se takto:
\begin{verbatim}
(modify (in :object snack :location fridge)
        (in :object snack :location schoolbag))
\end{verbatim}
Toto makro je ve skutečnosti jen zkratkou pro postupné volání \verb|assert|
a \verb|retract|. Neslouží ani tak k~ušetření zdrojového kódu, jako spíš
ke zčitelnění aktivací pravidla - z~jeho použití je evidentní vztah dvou
faktů.

\newpage
%%%%%%%%%%%%%%%%%%%%%%%%%%%%%%%%%%%%%%%%%%%%%%%%%%%%%%%%%%%%%%%%%%%%%%%%%%%%%%%%

\section{Použité algoritmy}
Přiložená implementace expertního systému zahrnuje několik z~literatury
čerpaných algoritmů. V~prvé řadě je to algoritmus Rete, který se stará
o~zjišťování, která odvozovací pravidla jsou splněna fakty ze znalostní
baze. Poté je použito několik velice jednoduchých algoritmů pro výběr
pravidla, jenž bude jako další aktivováno. 

\newpage
%%%%%%%%%%%%%%%%%%%%%%%%%%%%%%%%%%%%%%%%%%%%%%%%%%%%%%%%%%%%%%%%%%%%%%%%%%%%%%%%

\section{Popis implementace}

\end{document}
