\section{Teoretická část}
\subsection{Expertní systém}

Expertní systém je počítačový program, který usuzuje na základě znalostí
z~nějaké \emph{problémové domény} za účelem vyřešení zadaného problému nebo
poskytnutí odpovědi na otázku \cite{introduction}. Problémovou doménou zde
rozumíme množinu znalostí, omezenou na určitý okruh, který nás zajímá, s
vyloučením všech ostatních, irelevantních znalostí \cite{problem-domain}.
Znalosti z problémové domény jsou reprezentovány jejím
\emph{modelem}\footnote{\url{http://en.wikipedia.org/wiki/Domain\_model}}.
Expertní systém tedy pracuje s modelem zvolené problémové domény.

K typickým problémům, řešeným pomocí expertních systému patří
\cite{introduction}
\begin{itemize}
  \item plánování posloupnosti akcí vedoucích k zadanému cíli (např.
    plánování rozvrhu hodin se zohledněním obsazení učeben a vyučujících),
  \item diagnóza poruch systému a plánování jejich oprav, diagnóza a plánování
    léčby pacientů,
  \item konfugurace komplexních objektů (např. navržení serverového cloudu pro
    poskytnutí zadaných služeb),
  \item interpretace předem klasifikovaných dat (např. detekce nebezpečí z dat
    získaných z termografické kamery), analýza složení (vyhodnocení dat
    získaných metodami analytické chemie).
\end{itemize}

V první kapitole své bakalářské práce \cite{bakalarka} uvádím typické vlastnosti
expertních systémů, jejich zařazení v rámci informatiky a umělé inteligence a
jejich rozdíly oproti jiným typům programů, které daný problém mohou řešit.
Stěžejní charakteristikou expertního systému je úplné oddělení reprezentace
znalostí, nad kterými systém usuzuje, od samotného \emph{odvozovacího aparátu}.
Odvozovací aparát je tedy obecný a je schopen usuzovat nad jakoukoli problémovou
doménou, jejíž model lze v expertním systému reprezentovat.

Jak vyplývá z uvedené definice, expertní systém je hotový program schopný
odpovídat na dotazy, či řešit problémy v rámci zvolené problémové domény. Tento
program můžeme vytvořit od základu v nějakém obecném programovacím jazyce. To je
ale poměrně složité a pokud nemáme specifické nároky, nevyplatí se to. Častěji
použijeme existující \emph{nástroj} či \emph{knihovnu pro tvorbu expertních
systémů}. Tento nástroj už obsahuje odvozovací aparát a definuje reprezentaci
znalostí a~dotazů a~její syntax (viz kapitola \ref{knowledge representation}).
Poskytuje tedy tzv. \emph{prázdný expertní systém}. Takovým nástrojem je i
knihovna ExiL, která je přílohou této práce.

\clearpage
Zařazení expertního systému do procesu sestává z následujících fází:
\begin{enumerate}
  \item sestavení modelu zvolené problémové domény,
  \item výběr nástroje pro tvorbu expertních systémů podle typu domény a dotazů,
    které chceme být schopni zadávat,
  \item reprezentace modelu problémové domény použitím syntaxe zvoleného
    nástroje.
\end{enumerate}
Vytvořená reprezentace modelu problémové domény tvoří po načtení nástrojem pro
tvorbu expertních systémů tzv. \emph{znalostní bázi} výsledného expertního systému.
Tomu pak můžeme zadávat dotazy buď přímo, použitím jeho syntaxe, nebo skrze
nějaké uživatelské rozhraní. Je-li zvolený nástroj navržen jako knihovna, můžeme
také vytvořit další program, který zadává znalosti do expertního systému a/nebo
mu zadává dotazy a odpovědi na ně pak dále zpracovává.

Jednou z běžných vlastností expertních systémů je také schopnost vysvětlit, jak
systém k řešení problému dospěl. To umožňuje evaluaci správnosti řešení
a~poskytuje tak zpětnou vazbu pro případnou opravu znalostní báze, zadaného cíle
či dotazu, nebo odvozovacího aparátu (v případě vlastního návrhu expertního
systému).

Expretní systém nelze použít k řešení jakéhokoli typu problému. To je dáno
reprezentací znalostí v expertním systému a způsobem, jakým systém nad těmito
znalostmi usuzuje. Vlastnosti znalostí, které lze v expertním systému
reprezentovat, jejich reprezentace, způsoby zadávání cílů a dotazů i to, jak
systém zadaných cílů dosahuje, je náplní následujících kapitol.

\subsection{Reprezentace znalostí}

\begin{framed}
  \begin{itemize}
    \item vlastnosti znalostí zpracovatelných ES
      \begin{itemize}
        \item relevantní
        \item organizovaná (postihuje souvislosti mezi pojmy), extensively
          indexed and content-addressable (should enable cross-referencing)
        \item jednoznačná - i v reprezentaci, která umožnuje rozlišení můžeme
          znalost vyjádřit nejednoznačně
        \item self-contained - can't assume prior knowledge
      \end{itemize}
    \item reprezentace
      \begin{itemize}
        \item "set of syntactic and semantic conventions that make it possible
          to describe things", things ~ knowledge ~ state of the system
          (objects, their properties and relationships)
        \item syntax = "set of rules for combining symbols to form expressions
          in the representation language"
        \item semantics = how expressions so constructed should be interpreted
          - what is the meaning of the forms - typically by assigning meanings
          to individual symbols, than recurently inducing meaning of more
          complex expressions
        \item formally described, well-defined syntax and semantics
        \item unambiguous
        \item processable and storable by computer
        \item logical adequacy - representation should be capable of making all
          distinctions that you want to make
        \item heuristic power - there must be a way of using this representation
          to solve problems - the more expressive language (more possible
          distinctions), the the more difficult to manipulate during inference
          item natational convenience - expressions easy to write, read and
          understand (without knowing, how the computer will interpret them)
        \item declarative - descriptive $\rightarrow$ can be understood without
          knowing what states the program will go though when interpreting the
          representation
        \item example schemes - production rules, structured objects (e.g.
          templates), logic programs
        \item representation language - oriented towards organizing descriptions
          of objects and ideas, rather than stating sequences of instructions or
          storing simple data elements
      \end{itemize}
    \item problem solving involves reasoning about actions and states of the
      world
    \item problems can be formulated in terms of initial state, goal state
      and a set of operations, that can be employed in an attempt to
      transform one state into another
    \item symbolické výpočty - non-numeric computations, manipulation of
      symbolic representations
    \item reprezentace stavu pomocí symbolických struktur, inference jako jejich
      manipulace
    \item fakty a odvozovací pravidla
  \end{itemize}
\end{framed}

\subsection{Systém CLIPS}

Prázdný expertní systém CLIPS poskytuje obecnější a tudíž flexibilnější
reprezentaci faktů a odvozovacích pravidel než systém STRIPS. Atomy faktů CLIPSu
mohou být nejen symboly, ale také řetězce znaků a čísla (celá i desetinná).
Systém rozlišuje dva typy faktů - jednoduché a strukturované. Jednoduchý fakt je
reprezentován seznamem (uspořádanou \emph{n}-ticí) atomů, např.
\verb|(in box A)|. Strukturovaný fakt je reprezentován objektem s pojmenovanými
atributy (sloty), např. \verb|(in (object box) (location A))|. Hodnotami slotů jsou
opět atomy.

Vzory mají v CLIPSu stejný tvar jako fakty. Symboly proměnných zde začínají
znakem \verb|?|, např. \verb|?location|. Kongruenci faktů a vzorů lze definovat
podobně jako ve STRIPSu. U jednoduchých faktů srovnáváme atomy na odpovídajících
pozicích, u strukturovaných pak hodnoty odpovídajících slotů objektu.

Systém CLIPS umožňuje spojovat podmínky odvozovacích pravidel logickými
konjunkcemi i disjunkcemi a tyto libovolně vnořovat, navíc můžeme některé dílčí
podmínky negovat. Dále je možné podmínky pravidel kvantifikovat jak existenčně,
tak všeobecně. Podmínky CLIPSového odvozovacího pravidla tedy mohou vypadat
například takto \cite{clips}:
\begin{minted}{cl}
(or (and (temp high)
         (valve closed))
    (and (temp low)
         (valve open))).
\end{minted}

Definice pravidla CLIPSu vypadá například takto \cite{clips}:
\begin{minted}{cl}
(defrule system-flow
  (error-status ?status)
  (or (and (temp high)
           (valve closed))
      (and (temp low)
           (valve open)))
  =>
  (retract (error-status ?status))
  (assert (error-status confirmed))
  (printout t "The system is having a flow problem." crlf)).
\end{minted}
Definice obsahuje název pravidla, jeho podmínky a důsledky. Podmínky jsou od
důsledků odděleny symbolem \verb|=>|. Důsledky pravidla nejsou pouze vzory faktů
k~přidání do či odebrání ze znalostní báze, nýbrž libovolné výrazy jazyka, který
CLIPS definuje. Ty jsou při aktivaci pravidla, po nahrazení proměnných jejich
vazbami, vyhodnoceny CLIPSovým interpreterem.

Systém CLIPS poskytuje velmi široké možnosti, další už nebudu uvádět. Lze je
však najít v dokumentaci
systému\footnote{\url{http://clipsrules.sourceforge.net/OnlineDocs.html}}.
Praktická část této práce popisuje implementaci knihovny ExiL, která je
výsledkem práce a implementuje prázdný expertní systém inspirovaný právě
systémem CLIPS.  Ve srovnání s ním však poskytuje jen velmi omezené možnosti.
Knihovna je vytvořena v programovacím jazyce Common Lisp a v důsledcích pravidla
lze použít libovolné lispové výrazy, včetně volání maker pro modifikaci stavu
systému.


\subsection{Inference}
\begin{framed}
  \begin{itemize}
    \item výpočet jako posloupnost stavů s přechody danými aplikací pravidel
    \item prohledávání prostoru stavů (stromy)
    \item combinatorial explosion $\rightarrow$ heuristiky (strategie)
    \item cíle v STRIPS jako vzory
    \item dopředné vs. zpětné řetězení
    \item zpětné - means-ends analysis
    \item STRIPS x CLIPS
    \item reprezentace problému - počáteční stav, přechody, koncový stav,
      příp. cíle (zpětná inference)
    \item reprezentace problému
      \begin{itemize}
        \item problem solving involves reasoning about actions and states of the
          world (state = current knowledge = describes objects, their properties and
          relationonships)
        \item problems can be formulated in terms of initial state, goal state
          and a set of operations, that can be employed in an attempt to
          transform one state into another
        \item symbolické výpočty - non-numeric computations, manipulation of
          symbolic representations
      \end{itemize}
  \end{itemize}
\end{framed}

Expertní systémy rozdělujeme do dvou skupin podle toho, jakým způsobem nad
zadanými znalostmi usuzují. \textbf{Expertní systém se dopředným řetězením}
(inferencí) postupně vyvozuje závěry ze zadaných znalostí hledáním odvozovacích
pravidel se splněnými podmínkami. \textbf{Expertní systém se zpětným řetězením}
(inferencí) postupuje od zadaného cíle a hledá odvozovací pravidla, která vedou
k jeho splnění. Jejich podmínky pak zpracovává jako dílčí cíle.

Požadovaný cíl či dotaz může být součástí znalostí předaných systému (typické
pro systémy s dopřednou inferencí), nebo může být zadán samostatně (systémy se
zpětnou inferencí). Může nás také zajímat množina všech závěrů, odvoditelných ze
zadaných znalostí.

Zadáním cíle se systému dotazujeme, zda existuje posloupnost aplikace
odvozovacích pravidel, vedoucí k jeho splnění. V případě nale

Buď konkrétní cíl.
Cíl problému můžeme zadat systému jako dotaz (typické pro, ale ne omezené na,
zpětné řetězení).
Nebo nás může zajímat množina všech výsledků odvoditelných z báze.
poskytnutí odpovědi na otázku - zda lze ze znalostní báze odvodit fakt daného
tvaru (vzor), nalézt možné vazby proměnných

\subsection{Interpretace programu}
\begin{framed}
  \begin{itemize}
    \item interpretace výstupu programu
    \item plánování akcí
    \item hledání odpovědí
    \item dokazování vyplývání - sémantika podmínek
  \end{itemize}
\end{framed}

% \begin{framed}
%   \begin{itemize}
%     \item rozebrat, jak řeší řeší GPS v PoAIP problémy se zpětným řetězením
%     \item MYCIN, jakožo ES se zpětným řetězením, neřeší negativní znalost vůbec
%     \item co se týče kompozitních podmínek, používá MYCIN and-or stromy, šlo by
%       aplikovat v rete?
%     \item CLIPS basic programming guide - defrule construct - conflict resolution
%       strategies, LHS conditional elements
%   \end{itemize}
% \end{framed}
