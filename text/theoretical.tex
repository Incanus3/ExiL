\section{Teoretická část}

\subsection{Expertní systémy}
\begin{framed}
  \begin{itemize}
    \item co je expertní systém - vlastnosti, distinkce, typické problémy
  \end{itemize}
\end{framed}

\subsection{Reprezentace znalosti}
\begin{framed}
  \begin{itemize}
    \item pracuje se znalostí - vlastnosti znalosti zpracovatelné ES
    \item symbolická reprezentace znalosti $\rightarrow$ symbolické výpočty
      \item reprezentace stavu pomocí symbolických struktur, inference jako jejich manipulace
  \end{itemize}
\end{framed}

\subsection{Základní principy}
\begin{framed}
  \begin{itemize}
    \item rule-based systémy
    \item prohledávání prostoru stavů (stromy)
    \item combinatorial explosion $\rightarrow$ heuristiky (strategie)
  \end{itemize}
\end{framed}

\subsection{Dopředná a zpětná inference}
\begin{framed}
  \begin{itemize}
    \item dopředné vs. zpětné řetězení
    \item zpěnté - means-ends analysis
  \end{itemize}
\end{framed}

\subsection{Interpretace programu}
\begin{framed}
  \begin{itemize}
    \item interpretace výstupu programu
    \item plánování akcí
    \item hledání odpovědí
    \item dokazování vyplývání - sémantika podmínek
  \end{itemize}
\end{framed}

\subsection{Aplikace}
\begin{framed}
  \begin{itemize}
    \item strips, gps - zpětné řetězení
    \item mycin - složitější pravidla
    \item clips - dopředné řetězení
  \end{itemize}
\end{framed}

% \begin{framed}
%   \begin{itemize}
%     \item rozebrat, jak řeší řeší GPS v PoAIP problémy se zpětným řetězením
%     \item MYCIN, jakožo ES se zpětným řetězením, neřeší negativní znalost vůbec
%     \item co se týče kompozitních podmínek, používá MYCIN and-or stromy, šlo by
%       aplikovat v rete?
%     \item CLIPS basic programming guide - defrule construct - conflict resolution
%       strategies, LHS conditional elements
%   \end{itemize}
% \end{framed}
