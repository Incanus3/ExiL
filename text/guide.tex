%%%%%%%%%%%%%%%%%%%%%%%%%%%%%%%%%%%%%%%%%%%%%%%%%%%%%%%%%%%%%%%%%%%%%%%%%%%%%%%%
\subsection{Uživatelská příručka}
\subsubsection{Základní pojmy}

Nyní stručně zadefinuji základní pojmy, nutné pro pochopení fungování knihovny
ExiL a práci s ní. Význam pojmů bude jasnější, jakmile si je ukážeme na
příkladech. K těmto pojmům se posléze vrátím i~v~teoretické části textu
a~jejich popis rozšířím o další souvislosti.

První dva pojmy staví na pojmu znalost, který chápeme intuitivně, nikoli na
následujícím pojmu znalost, jak jej chápeme v~ExiLu (v takovém případě by byla
definice cyklická).

Pojem expertního systému zatím chápejme tak, jak jsem jej představil v úvodu
práce. V teoretické části rozeberu pojem v potřebné šíři.
\begin{description}[leftmargin=5cm,style=sameline,align=right,labelsep=0.5cm]
  \item[fakt] elementární statická znalost - tvrzení
  \item[(odvozovací) pravidlo] elementární odvozovací znalost ve formě implikace
  \item[znalost (v ExiLu)] množina faktů a pravidel
  \item[znalostní báze] výchozí znalost expertního systému
  \item[pracovní paměť] aktuální množina faktů
  \item[inference] odvozování - postupná aplikace pravidla modus
    ponens\footnote{\url{http://en.wikipedia.org/wiki/Modus_ponens}} na
    konjunkci faktů a odvozovací pravidlo
\end{description}
Pojem \emph{pracovní paměť} není příliš intuitivní. Jde o doslovný překlad
v~literatuře užívaného pojmu \emph{working memory}, kterým je označována množina
faktů (tvrzení), které expertní systém v danou chvíli považuje za platné. Nejde
tedy ve skutečnosti o paměť, nýbrž o obsah pomyslné paměti. Pojem pracovní
množina faktů by byl jistě výstižnější, bohužel ale také značně těžkopádný.

%%%%%%%%%%%%%%%%%%%%%%%%%%%%%%%%%%%%%%%%%%%%%%%%%%%%%%%%%%%%%%%%%%%%%%%%%%%%%%%%
\input{guide-01-structure}
\input{guide-02-knowledge-base}
\input{guide-03-working-memory}
\input{guide-04-inference}
%%%%%%%%%%%%%%%%%%%%%%%%%%%%%%%%%%%%%%%%%%%%%%%%%%%%%%%%%%%%%%%%%%%%%%%%%%%%%%%%
\subsubsection{Sledování průběhu inference}
\label{inference tracing}
ExiL umožnuje sledovat několik typů událostí, ke kterým dochází během inference.
K nastavení sledovaných událostí slouží makra \verb|watch| a \verb|unwatch|. K
zjištění stavu sledování pak makro \verb|watchedp|.

Základním výstupem programu \ref{typical structure} na straně \pageref{typical
structure} je
\begin{minted}{cl}
Firing MOVE
Firing PUSH
Firing STOP
Halting.
\end{minted}
Zapneme-li sledování faktů voláním \verb|(watch facts)|, obdržíme výstup
\begin{minted}{cl}
==> (IN BOX A)
==> (IN ROBOT B)
==> (GOAL MOVE BOX A B)
Firing MOVE-ROBOT
<== (IN ROBOT B)
==> (IN ROBOT A)
Firing MOVE-OBJECT
<== (IN ROBOT A)
<== (IN BOX A)
==> (IN ROBOT B)
==> (IN BOX B)
Firing STOP
Halting.
\end{minted}

Sledování pravidel - \verb|(watch rules)| - přidává informace o pravidlech
přidaných do (odebraných ze) znalostní báze, například
\begin{minted}{cl}
==> (RULE STOP
  (GOAL MOVE ?OBJECT ?FROM ?TO)
  (IN ?OBJECT ?TO)
  =>
  (HALT)).
\end{minted}

Po zapnutí sledování agendy (voláním \verb|(watch activations)|, název je kvůli
kompatibilitě se systémem CLIPS) budeme navíc informováni o shodách, které do
agendy přibyly, nebo z ní byly odstraněny. Výstup programu pak bude následující:
\begin{minted}[samepage]{cl}
==> (MATCH MOVE-ROBOT
      ((GOAL MOVE BOX A B) (IN BOX A) NIL (IN ROBOT B)))
Firing MOVE-ROBOT
==> (MATCH MOVE-OBJECT
      ((GOAL MOVE BOX A B) (IN BOX A) (IN ROBOT A)))
==> (MATCH MOVE-ROBOT
      ((GOAL MOVE BOX A B) (IN BOX A) NIL (IN ROBOT A)))
<== (MATCH MOVE-ROBOT
      ((GOAL MOVE BOX A B) (IN BOX A) NIL (IN ROBOT A)))
Firing MOVE-OBJECT
==> (MATCH STOP
      ((GOAL MOVE BOX A B) (IN BOX B)))
Firing STOP
Halting
\end{minted}
Každá shoda je zde reprezentována názvem pravidla a posloupností faktů, které
byly spárovány s jeho podmínkami. Odtud můžeme snadno odvodit substituci, jež
byla při vyhodnocení použita. Negované podmínky nejsou spárovány s žádným
konkréním faktem, proto jsou na odpovídajících pozicích hodnoty \verb|NIL|.

Je zde také vidět, že po aktivaci pravidla \verb|move-robot| se v agendě na
chvíli objeví opětovná shoda tohoto pravidla. To je způsobeno tím, že obsah
agendy se přepočítává po každé změně pracovní paměti, takže se zde mohou objevit
dočasné výsledky.

%%%%%%%%%%%%%%%%%%%%%%%%%%%%%%%%%%%%%%%%%%%%%%%%%%%%%%%%%%%%%%%%%%%%%%%%%%%%%%%%
\subsubsection{Undo/redo}

Jedním z implementovaných rozšíření původního programu je schopnost vrácení
provedených změn. K tomu slouží makra \verb|undo| a \verb|redo|. Ta lze použít k
vrácení jakékoli akce s vedlejším efektem, včetně kroků inference. K vypsání
zásobníků s~akcemi, které je možné vrátit, jsou k dispozici makra
\verb|undo-stack| a \verb|redo-stack|.

Pokud například vyhodnotíme prvních 32 řádků programu \ref{typical structure} na
straně \pageref{typical structure} a~zavoláme dvakrát \verb|undo|, bude výpis
zásobníků následující (přeformátováno):
\begin{minted}[samepage]{cl}
EXIL-USER> (undo-stack)
  1: (defrule MOVE-ROBOT
       ((GOAL MOVE ?OBJECT ?FROM ?TO)
        (IN ?OBJECT ?FROM)
        (- IN ROBOT ?FROM) (IN ROBOT ?Z)
        =>
        (RETRACT (IN ROBOT ?Z))
        (ASSERT (IN ROBOT ?FROM))))
  2: (deffacts WORLD
       ((IN BOX A) (IN ROBOT B) (GOAL MOVE BOX A B)))
\end{minted}
\begin{minted}[samepage]{cl}
EXIL-USER> (redo-stack)
  1: (defrule MOVE-OBJECT
       ((GOAL MOVE ?OBJECT ?FROM ?TO)
        ?OBJ-POS <- (IN ?OBJECT ?FROM)
        ?ROB-POS <- (IN ROBOT ?FROM)
        =>
        (RETRACT ?ROB-POS)
        (RETRACT ?OBJ-POS)
        (ASSERT (IN ROBOT ?TO))
        (ASSERT (IN ?OBJECT ?TO))))
  2: (defrule STOP
       ((GOAL MOVE ?OBJECT ?FROM ?TO)
        (IN ?OBJECT ?TO)
        =>
        (HALT)))
\end{minted}
Vidíme tedy, že jsme vrátili zpět definice pravidel \verb|move-object| a
\verb|stop| (ty bychom mohli opět provést voláním \verb|(redo)|). Dalším voláním
\verb|(undo)| by pak byla vrácena definice pravidla \verb|move-robot| a poté
definice skupiny faktů \verb|world|.

Nemá-li akce žádný vedlejší efekt - např. volání \verb|assert| s faktem, který
už v~pracovní paměti je, či volání \verb|run| ve chvíli, kdy už není co
odvozovat - prázdná akce se na zásobník neuloží.

%%%%%%%%%%%%%%%%%%%%%%%%%%%%%%%%%%%%%%%%%%%%%%%%%%%%%%%%%%%%%%%%%%%%%%%%%%%%%%%%
\subsubsection{Zpětná inference}

Dalším z implementovaných rozšíření je možnost zpětné inference. Inference
popsaná v sekci \ref{inference} je dopředná. V každém kroku jsou nalezeny
všechny možnosti dalšího postupu odvozování, načež je zvolena jedna, kterou se
program dále ubírá. To činí průběh inference značně nedeterministickým. Možnosti
postupu, které nebyly vybrány, mohou být navíc dalším postupem ztraceny, pokud
aktivace některého pravidla zneplatní podmínky jiného.
Míru nedeterminismu můžeme snížit tím, že budeme navrhovat odvozovací pravidla
tak, aby se výpočet neubíral nechtěnými cestami. To ale není vždy jednoduché,
nebo dokonce možné.

Zpětná inference naproti tomu umožňuje definovat cíle, kterých chceme dosáhnout.
K tomu slouží makro \verb|defgoal|, kterému předáme vzor ve stejném formátu,
jako u podmínek pravidel. Definice cíle ovšem nepodporuje negaci ani navázání
faktu na proměnnou (k tomu ani není důvod). Cíle je pak možné vypsat voláním
\verb|(goals)|. Cíl můžeme také odebrat makrem \verb|undefgoal|.
Ke spuštění zpětné inference slouží funkce \verb|back-step| a
\verb|back-run|, podobně jako u inference dopředné.

Mějme následující znalostní bázi:
\begin{minted}{cl}
(deffacts world
  (have-money))

(defrule buy-car
  (have-money)
  =>
  (retract (have-money))
  (assert (have-car)))

(defrule pay-rent
  (have-money)
  =>
  (retract (have-money))
  (assert (rent-payed))).
\end{minted}
Spustíme-li dopřednou inferenci, systém nám vesele doporučí nákup auta.
Mít nové auto je sice pěkné, hrozí-li nám ale vyhození z pronajatého bytu,
není nákup auta pravděpodobně cestou, kterou bychom se chtěli ubírat.
Systém by nám v tuto chvíli mohl stejně dobře doporučit správnou cestu. Že bylo
vybráno zrovna první pravidlo je výsledkem toho, jak funguje síť RETE, která
pravidla vyhodnocuje. Za daných okolností ale nechceme špatnou variantu ani
připouštět.

V tomto případě bychom mohli upravit definici programu tak, že do znalostní báze
přidáme informaci o cíli, kterou budou pravidla zohledňovat, podobně jako
v~příkladu \ref{typical structure} na straně \pageref{typical structure}. Muset
ale programovat zohlednění cíle v každém pravidle je přinejmenším otravné. U
větších programů to navíc může být velmi náročné, neboť cíl bude třeba
programově modifikovat v průběhu výpočtu.

S použitím zpětné inference je problém podstatně jednodušší. Zavoláme-li
\begin{minted}{cl}
(reset)
(defgoal (rent-payed))
(back-run),
\end{minted}
bude výsledkem výstup
\begin{minted}[samepage]{cl}
All goals have been satisfied
(RENT-PAYED) satisfied by (RULE PAY-RENT
  (HAVE-MONEY)
  =>
  (RETRACT (HAVE-MONEY))
  (ASSERT (RENT-PAYED)))
(HAVE-MONEY) satisfied by (HAVE-MONEY).
\end{minted}
Zde vidíme, že po spuštění zpětné inference nezačal systém bezhlavě provádět
akce, ke kterým měl dostatečné prostředky. Místo toho systém uvážil zadaný cíl a
jal se hledat akce, které k jeho splnění směřují.

Uvažme nyní složitější příklad (definice šablon vynechána):
\begin{minted}[samepage]{cl}
(deffacts world
  (female jane)
  (male john)
  (parent :parent jane :child george)
  (parent :parent john :child george))

(defrule father-is-male-parent
  (male ?father)
  (parent :parent ?father :child ?child)
  =>
  (assert (father :father ?father :child ?child)))

(defrule mother-is-female-parent
  (female ?mother)
  (parent :parent ?mother :child ?child)
  =>
  (assert (mother :mother ?mother :child ?child)))
\end{minted}
Zajímá-li nás, kdo je matkou George, můžeme zkusit spustit dopřednou inferenci.
Po jejím skončení bude v pracovní paměti jak informace o Georgově matce, tak
o~jeho otci. Systém se tedy v tomto případě dobral správného výsledku, vypočítal
ale i další fakty, které nás nezajímaly. Dokážeme si snadno představit, že ve
větším programu může být výpočet všech odvoditelných závěrů velmi výpočetně
a tudíž i časově náročný.

Spustíme-li naopak zpětnou inferenci voláním
\begin{minted}[samepage]{cl}
(reset)
(defgoal (mother :mother ?mother-of-george :child george))
(back-run),
\end{minted}
je výsledkem
\begin{minted}[samepage]{cl}
All goals have been satisfied
(MOTHER (MOTHER . ?MOTHER-OF-GEORGE) (CHILD . GEORGE)) satisfied by
  (RULE MOTHER-IS-FEMALE-PARENT
    (FEMALE ?MOTHER)
    (PARENT (PARENT . ?MOTHER) (CHILD . ?CHILD))
    =>
    (ASSERT (MOTHER :MOTHER ?MOTHER :CHILD ?CHILD)))
(FEMALE ?MOTHER) satisfied by (FEMALE JANE)
(PARENT (PARENT . JANE) (CHILD . GEORGE)) satisfied by
  (PARENT (PARENT . JANE) (CHILD . GEORGE))
These variable bindings have been used:
((?MOTHER-OF-GEORGE . JANE))
\end{minted}
Systém tedy vyvodil pouze závěr, který nás zajímal.

Zpětná inference umožnuje také výpočet alternativních odpovědí, tedy hledání dalších
cest výpočtu (a vazeb proměnných), které vedou ke splnění všech cílů. Na další
alternativní odpověď se dotážeme jednoduše opětovným voláním \verb|(back-run)|.

Zajímají-li nás například oba rodiče George, můžeme zadat cíl
\cl|(defgoal (parent :parent ?parent :child george)).|
Pokud poté třikrát zavoláme \verb|(back-run)|, obdržíme výstup
\begin{minted}[samepage]{cl}
All goals have been satisfied
(PARENT (PARENT . ?PARENT) (CHILD . GEORGE)) satisfied by
  (PARENT (PARENT . JANE) (CHILD . GEORGE))
These variable bindings have been used:
((?PARENT . JANE))

All goals have been satisfied
(PARENT (PARENT . ?PARENT) (CHILD . GEORGE)) satisfied by
  (PARENT (PARENT . JOHN) (CHILD . GEORGE))
These variable bindings have been used:
((?PARENT . JOHN))

No feasible answer found.
\end{minted}
Systém tedy najde obě možné odpovědi (vazby proměnných), vedoucí ke splnění
cíle, načež oznámí, že další odpověď už neexistuje.

Zpětná inference nemění obsah pracovní paměti. To ani není možné vzhledem
k tomu, že ve chvíli, kdy inference zvolí pravidlo, jehož důsledky vedou ke
splnění aktuálního cíle, nejsou jeho podmínky často ještě splněny. Místo toho
pracuje zpětná inference pouze s množinou cílů.

Jednotlivé cíle jsou postupně vybírány a~hledají se cesty k jejich splnění.
Inference nejprve zkoumá fakty v~pracovní paměti. Není-li aktuální cíl splněn
žádným z platných faktů, uvažuje dále jednotlivá pravidla. Najde-li pravidlo,
které by po aktivaci vedlo ke splnění aktuálního cíle, naváže proměnné, použité
v jeho důsledcích, podle vzoru cíle. Tyto vazby poté aplikuje na jeho podmínky
(tedy opačně než u inference dopředné) a ty pak přidá do množiny cílů. Použité
vazby proměnných se navíc aplikují i na zbytek cílů.

Zkusme nyní krokovat (použitím \verb|(back-step)|) předchozí příklad s dotazem na
matku George s průběžným výpisem cílů pomocí \verb|(goals)|.
\begin{minted}[samepage]{cl}
GOALS: ((MOTHER :MOTHER ?MOTHER-OF-GEORGE :CHILD GEORGE))
(MOTHER (MOTHER . ?MOTHER-OF-GEORGE) (CHILD . GEORGE)) satisfied by
  (RULE MOTHER-IS-FEMALE-PARENT
    (FEMALE ?MOTHER)
    (PARENT (PARENT . ?MOTHER) (CHILD . ?CHILD))
    =>
    (ASSERT (MOTHER :MOTHER ?MOTHER :CHILD ?CHILD)))

GOALS: ((FEMALE ?MOTHER) (PARENT :PARENT ?MOTHER :CHILD GEORGE))
(FEMALE ?MOTHER) satisfied by (FEMALE JANE)

GOALS: ((PARENT :PARENT JANE :CHILD GEORGE))
(PARENT (PARENT . JANE) (CHILD . GEORGE)) satisfied by
  (PARENT (PARENT . JANE) (CHILD . GEORGE))
\end{minted}
V prvním kroku je proměnná \verb|?mother-of-george|, použitá v definici cíle,
nahrazena proměnnou \verb|?mother|, použitou v důslecích pravidla
\verb|mother-if-female-parent|. Proměnná \verb|?child| v důsledcích je navázána
na \verb|george| a touto vazbou jsou nahrazeny výskyty proměnné v podmínkách
pravidla. Podmínky jsou poté přidány do množiny cílů. Původní cíl je poté
z~množiny odstraněn.

V druhém kroku je nový cíl \verb|(female ?mother)| porovnán s faktem
\verb|(female jane)| v pracovní paměti, je jím splněn a v posledním cíli je
proměnná \verb|?mother| navázána na \verb|jane|. Tento cíl je pak
v posledním kroku triviálně splněn identickým faktem.

Aktuální stav výpočtu pomocí zpětné inference je ztracen, zavoláme-li během
krokování \verb|defgoal| nebo \verb|undefgoal|. Kdyby tomu tak nebylo, mohli
bychom systém uvést do nekonzistentního stavu.

Implementace zpětné inference v ExiLu je velmi omezená. V důsledích pravidel
zohledňuje pouze specifikace faktů ve voláních \verb|assert|. Nelze ji tedy
aplikovat u pravidel, která fakty z pracovní paměti odstraňují, či je
modifikují. Inference také neumí pracovat s negovanými cíli, tudíž ani s
pravidly, která mají negované podmínky (neboť tyto by při použití pravidla
mezi cíli objevily).

Použití zpětné inference nemusí nutně snížit míru nedeterminismu výpočtu.
Existuje-li několik pravidel, která vedou ke splnění aktuálního cíle, bude
výpočet opět nedeterministický. Na rozdíl od dopředné inference lze ale
alternativní cesty výpočtu postupně prohledat. To je možné jednak díky
backtrackingu - nevede-li daná cesta ke splnění všech cílů, výpočet se vrátí a
zkusí se ubírat jinudy. Dále díky možnosti dotázat se na alternativní odpovědi.

\input{guide-08-cleanup}
%%%%%%%%%%%%%%%%%%%%%%%%%%%%%%%%%%%%%%%%%%%%%%%%%%%%%%%%%%%%%%%%%%%%%%%%%%%%%%%%
\subsubsection{Práce s více prostředími}
\label{multiple environments}
Při práci s ExiLem se nemusíme omezovat pouze na jedno prostředí (byť si s ním
často vystačíme). Nové prostředí lze definovat voláním \verb|defenv| s~názvem
prostředí, např. \verb|(defenv test)|. Prostředí pak lze přepnout voláním
\verb|(setenv test)|, případně smazat voláním \verb|(undefenv test)|.

Každé prostředí má oddělený stav, tedy trvalé i dočasné hodnoty (viz předchozí
sekce). Na název aktuálního prostředí se lze dotázat voláním
\verb|(current-environment)|. Název výchozího prostředí je \verb|default|.
Seznam všech prostředí získáme voláním \verb|(environments)|.

Máme-li už definované prostředí, např. \verb|test|, opětovné volání
\verb|(defenv test)| skončí výjimkou. Tím je zajištěno, že si omylem nevymažeme
celé prostředí.  Chceme-li jej opravdu vymazat, musíme volat
\verb|(defenv test :redefine t)|.

%%%%%%%%%%%%%%%%%%%%%%%%%%%%%%%%%%%%%%%%%%%%%%%%%%%%%%%%%%%%%%%%%%%%%%%%%%%%%%%%
\subsubsection{Volání ExiLu z jiného kódu a naopak}

Doposud jsem v ukázkách práce s ExiLem pracoval většinou s makry. Byla to
následující:
\begin{description}[leftmargin=6.7cm,style=sameline,align=right,labelsep=0.5cm]
  \item[definice šablon] \verb|deftemplate undeftemplate| \verb|find-template|
  \item[definice skupin faktů] \verb|deffacts undeffacts find-fact-group|
  \item[definice pravidel] \verb|defrule undefrule find-rule|
  \item[modifikace pracovní paměti] \verb|assert retract modify|
  \item[definice cílů] \verb|defgoal undefgoal|
  \item[sledování průběhu inference] \verb|watch unwatch watchedp|
  \item[strategie výběru shody] \verb|defstrategy setstrategy|
  \item[definice prostředí] \verb|defenv undefenv setenv|
\end{description}
Tato makra berou jako parametry symboly a/nebo seznamy a tyto automaticky
\emph{quotují}. To je pohodlné, pracujeme-li s knihovnou přímo. Představme si
ale, že chceme knihovnu volat z jiného kódu a například specifikace faktů, které
předáváme makru \verb|assert|, generovat nějakou funkcí.

Protože makro \verb|assert| seznamy se specifikacemi faktů quotuje, místo aby je
vyhodnotilo, nelze ho v tomto případě použít. Výsledkem volání
\cl|(assert (generate-fact))| totiž bude přidání faktu \verb|(generate-fact)| do
pracovní paměti. K tomuto účelu poskytuje ExiL ke všem uvedeným makrům funkční
alternativy, které parametry vyhodnocují. Tyto jsou označeny suffixem \verb|f|,
například \verb|assertf|. Tímto suffixem sice v Lispu typicky označujeme
destruktivní makra, která mění svůj argument, v případě exilových maker ale
záměna nehrozí.

Dotazovací funkce a makra jako \verb|facts|, \verb|goals|,
\verb|find-fact-group|, apod. navíc nevypisují hodnoty na výstup, nýbrž vrací
externí reprezentaci objektů, se kterou je možné dále manipulovat a pak ji třeba
systému předat zpátky. To umožnuje například následující (nepříliš užitečné)
volání:
\begin{minted}{cl}
(dolist (fact (facts))
  (retractf fact)
  (assertf (cons 'my-fact fact))).
\end{minted}

Funkční alternativy jako \verb|assertf| také umožnují použití složitějších
konstruktů v důsledcích pravidla. Bude-li například podmínka pravidla
\begin{minted}{cl}
(defrule surround-by-as
  (palindrome ?p)
  =>
  (assertf `(palindrome ,(concatenate 'string "a" ?p "a"))))
\end{minted}
splněna faktem \verb|(palindrome "b")|, přibyde po jeho aktivaci do pracovní
paměti fakt \verb|(palindrome "aba")|. Funkci \verb|assertf| bohužel nelze
použít se zpětnou inferencí, neboť ta nemá šanci předvídat, jaký fakt bude
jejím voláním do pracovní paměti přidán.

V důsledcích pravidla bychom také mohli chtít například upozornit jinou část
programu na událost, ke které došlo. Protože ExiL nahrazuje výskyty
proměnných v celé důsledkové části pravidla, lze toho snadno dosáhnout. Je ale
třeba dát pozor na quotování hodnot proměnných. Uvažme pravidlo
\begin{minted}{cl}
(defrule move-robot
  (goal move robot ?from ?to)
  (in robot ?from)
  =>
  (retract (in robot ?from))
  (assert (in robot ?to))
  (notify 'moving-robot '?from '?to))
\end{minted}
a fakt \verb|(goal move robot A B)| v pracovní paměti. Kdybychom ve volání
\verb|notify| proměnné \verb|?from| a \verb|?to| nequotovali, volání by se při
aktivaci pravidla vyhodnotilo jako \verb|(notify 'moving-robot A B)|. To by
pravděpodobně skončilo chybovou hláškou sdělující, že proměnné \verb|A| a
\verb|B| nejsou definovány.

Makra \verb|deftemplate|, \verb|deffacts|, \verb|defrule| a \verb|modify| berou
specifikace slotů šablony, faktů, těla pravidla a změn k provedení (v tomto
pořadí) jako další parametry. Díky tomu nemusíme tyto parametry obalovat do
dalšího seznamu. Jejich funkční alternativy naproti tomu očekávají tyto
parametry v jednom seznamu, například
\cl|(deffactsf 'world (list '(in box A) '(in robot B))).|
To umožňuje snazší generování těchto specifikací funkcemi.

%%%%%%%%%%%%%%%%%%%%%%%%%%%%%%%%%%%%%%%%%%%%%%%%%%%%%%%%%%%%%%%%%%%%%%%%%%%%%%%%
\subsubsection{CLIPSová syntax}
\begin{framed}
  \begin{itemize}
    \item deftemplate, fact specifiery
    \item volání facts s číslem
    \item projít příručku clipsu a vzpomenout si, co v ExiLu ještě nebylo
    \item typy
    \item multisloty
  \end{itemize}
\end{framed}

Dalším z požadavků zadání práce bylo přiblížit syntax exilových volání systému
CLIPS, aby bylo možné programy v něm napsané snáze převést na programy exilové.
Toho se mi podařilo dosáhnout jen částečně.

Systém CLIPS použivá jiný formát specifikací slotů šablony, strukturovaných
faktů a požadovaných změn při volání \verb|modify|. Tuto syntax nyní ExiL podporuje
také. Příklad \ref{clips syntax} na straně \pageref{clips syntax} ukazuje
definici znalostní báze ekvivalentní příkladu \ref{structured facts} s použitím
CLIPSové syntaxe. Syntax je dostatečně odlišná na to, aby ji ExiL rozpoznal,
není tedy třeba syntaktický mód nijak přepínat. Díky tomu dokonce můžeme obě
syntaxe kombinovat.

ExiL také po vzoru CLIPSu umožňuje omezit seznam vrácený voláním \verb|(facts)|
volitelnými číselnými parametry. První volitelný parametr udává index prvního
faktu v seznamu (číslováno od 1). Druhý parametr udává index posledního faktu.
Třetí parametr pak maximální počet vrácených faktů.

Makra \verb|assert| a \verb|retract| také nyní umožňují přidání či odebrání
více faktů najednou. Makru \verb|retract| lze navíc místo specifikací faktů k
odstranění předat jejich číselné indexy v seznamu faktů. Obě možnosti lze
dokonce kombinovat.


\begin{listing}[t]
\caption{Definice znalostní báze s použitím CLIPSové syntaxe}
\label{clips syntax}
\begin{clcode}
(deftemplate goal
  (slot action (default move))
  (slot object)
  (slot from)
  (slot to))

(deftemplate in
  (slot object)
  (slot location))

(deffacts world
  (in (object robot) (location A))
  (in (object box) (location B))
  (goal (object box) (from B) (to A))).

(defrule move-robot
  (goal (action move) (object ?obj) (from ?from))
  (in (object ?obj) (location ?from))
  (- in (object robot) (location ?from))
  ?robot <- (in (object robot) (location ?z))
  =>
  (modify ?robot (location ?from)))

(defrule move-object
  (goal (action move) (object ?obj) (from ?from) (to ?to))
  ?object <- (in (object ?obj) (location ?from))
  ?robot <- (in (object robot) (location ?from))
  =>
  (modify ?robot (location ?to))
  (modify ?object (location ?to)))

(defrule stop
  ?goal <- (goal (action move) (object ?obj) (to ?to))
  (in (object ?obj) (location ?to))
  =>
  (halt))
\end{clcode}
\end{listing}

\FloatBarrier

%%%%%%%%%%%%%%%%%%%%%%%%%%%%%%%%%%%%%%%%%%%%%%%%%%%%%%%%%%%%%%%%%%%%%%%%%%%%%%%%
\subsubsection{Grafické uživatelské rozhraní}
\label{guide gui}

Pro prostředí LispWorks jsem k ExiLu implementoval minimalistické grafické
uživatelské rozhraní, zobrazené na obrázku \ref{gui} To sestává z hlavního okna
s deseti tlačítky organizovanými do tří řad (vpravo nahoře).

První řada tlačítek - \uv{Facts}, \uv{Templates}, \uv{Rules} a \uv{Agenda}
slouží k~zobrazení podoken s jednotlivými položkami. Okno \uv{Facts} zobrazuje
seznam faktů v pracovní paměti a umožňuje jejich odebrání tlačítkem
\uv{Retract fact}. Okno \uv{Templates} zobrazuje seznam definovaných šablon a
umožňuje jejich odebrání tlačítkem \uv{Undefine template}. Okno \uv{Rules}
zobrazuje seznam definovaných odvozovacích pravidel a umožňuje jejich odebrání
tlačítkem \uv{Undefine rule}. Poslední okno \uv{Agenda} zobrazuje seznam
aktuálních shod v agendě. Seznamy v~oknech se automaticky obnovují při každé
změně zobrazených hodnot.

Druhá řada tlačítek - \uv{Reset}, \uv{Step}, \uv{Run} a \uv{Halt} umožňuje
řízení inference. Tlačítka volají stejnojmenné funkce ExiLu.

Poslední řada tlačítek - \uv{Undo} a \uv{Redo} slouží k vracení a opětovnému
provedení akcí voláním stejnojmenných funkcí ExiLu.

\begin{figure}[h]
\includegraphics[width=\textwidth]{exil-gui.png}
\caption{Grafické uživatelské rozhraní}
\label{gui}
\end{figure}

Rozhraní lze zobrazit voláním \verb|(exil-gui:show-gui)|. Každé prostředí má
vlastní rozhraní. Volání \verb|show-gui| bez parametru zobrazí rozhraní k
aktuálnímu prostředí. Jako volitelný parametr můžeme funkci předat název
prostředí, jehož rozhraní chceme zobrazit. Máme-li tedy definováno více
prostředí, můžeme si ke každému z nich zobrazit uživatelské rozhraní. Po načtení
knihovny v prostředí LispWorks se automaticky zobrazí uživatelské rozhraní pro
výchozí prostředí.

