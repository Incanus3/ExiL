%%%%%%%%%%%%%%%%%%%%%%%%%%%%%%%%%%%%%%%%%%%%%%%%%%%%%%%%%%%%%%%%%%%%%%%%%%%%%%%%
\subsection{Uživatelská příručka}
\subsubsection{Základní pojmy}
\begin{framed}
  \begin{itemize}
    \item TODO: v teoretické části uvést pojmy znovu s citacemi
    \item TODO: production memory v exilu splývá s pravidly znalostní báze - na
      příhodném místě uvést, že v teorii a některých systémech se rozlišuje
    \item TODO: ke každému pojmu později jasně uvést příklad
    \item systém ve skutečnosti pracuje s \emph{reprezentací} znalostí, ale to
      nás zatím nezajímá
    \item dodetečné pojmy by mohly být definovány průběžně v boxech s odkazem do
      teoretické části
  \end{itemize}
\end{framed}

Nyní stručně zadefinuji základní pojmy, nutné pro pochopení fungování knihovny
ExiL a práci s ní. Význam pojmů bude jasnější, jakmile si je ukážeme na
příkladech. K těmto pojmům se posléze vrátím i~v~teoretické části textu
a~jejich popis rozšířím o další souvislosti.

První dva pojmy staví na pojmu znalost, který chápeme intuitivně a nebudu se jej
ani snažit definovat, nikoli na následujícím pojmu znalosti, jak ji chápeme
v~ExiLu (v takovém případě by byla definice cyklická).

Pojem expertního systému zatím chápejme tak, jak jsem jej představil v úvodu
práce. Pojem pochopitelně v praktické části rozeberu v potřebné šíři.
\begin{description}[leftmargin=6cm,style=sameline,align=right,labelsep=0.5cm]
  % \item[problémová doména] množina pojmů relevantních pro řešení určité skupiny
  %   problémů
  \item[fakt] elementární statická znalost - tvrzení
  \item[(odvozovací) pravidlo] elementární odvozovací znalost - pokud víme, že
    (ne)platí nějaká tvrzení, můžeme odvodit, že platí i~nějaká další
  \item[znalost (v ExiLu)] množina faktů a pravidel
  \item[znalostní báze] výchozí znalost expertního systému
  \item[pracovní paměť] aktuální množina faktů
  % \item[production memory] aktuální množina pravidel
  \item[inference] odvozování - postupná aplikace odvozovacích pravidel
\end{description}
Pojem \emph{pracovní paměť} není příliš intuitivní. Jde o doslovný překlad
v~literatuře užívaného pojmu \emph{working memory}, kterým je označována množina
faktů (tvrzení), které expertní systém v danou chvíli považuje za platné. Nejde
tedy ve skutečnosti o paměť, nýbrž o obsah pomyslné paměti. Pojem pracovní
množina faktů by byl jistě výstižnější, bohužel ale také značně těžkopádný.

% Pojmy working memory a production memory lze samozřejmě doslova přeložit, ale
% žádný z překladů mi nepřijde příliš názorný (zvlášť vzhledem k tomu, že ve
% skutečnosti nejde o paměť, nýbrž aktuální obsah pomyslné paměti), budu tedy
% používat anglická spojení užívaná v literatuře. Představa pojmů bude jasnější,
% jakmile si je ukážeme na příkladech.

%%%%%%%%%%%%%%%%%%%%%%%%%%%%%%%%%%%%%%%%%%%%%%%%%%%%%%%%%%%%%%%%%%%%%%%%%%%%%%%%
\subsubsection{Struktura programu}

\begin{framed}
  \begin{itemize}
    \item základní fáze vysokoúrovňově
    \begin{itemize}
      \item definice znalostní báze - fakta, pravidla - podmínky (patterny,
        proměnné), důsledky
      \item inicializace pracovní paměti ze znalostní báze
      \item inference - jen nastínit cyklus - vyhodnocení podmínek, výběr pravidla,
        aktivace
    \end{itemize}
    \item ke každému ze základních pojmů uvést příklad z programu.
    \item TODO: uvést výstup programu
  \end{itemize}
\end{framed}

\begin{listing}[h]
\caption{Základní struktura exilového programu}
\label{typical structure}
\begin{clcode}
;;; definition of knowledge base
;; facts
(deffacts world
  (in robot A)
  (in box B)
  (goal move box B A))

;; inference rules
(defrule move-robot
  (goal move ?object ?from ?to)
  (in ?object ?from)
  (- in robot ?from)
  (in robot ?z)
  =>
  (retract (in robot ?z))
  (assert (in robot ?from)))

(defrule move-object
  (goal move ?object ?from ?to)
  (in ?object ?from)
  (in robot ?from)
  =>
  (retract (in robot ?from))
  (assert (in robot ?to))
  (retract (in ?object ?from))
  (assert (in ?object ?to)))

(defrule stop
  ?goal <- (goal move ?object ?from ?to)
  (in ?object ?to)
  =>
  (retract ?goal)
  (halt))

;;; initialization of working memory
(reset)

;;; inference execution
(run)
\end{clcode}
\end{listing}

Příklad \ref{typical structure} (na straně \pageref{typical structure}) ukazuje
typickou strukturu programu nad knihovnou ExiL (dále exilový program). Tento
program používá strukturovaných faktů, začíná tudíž definicemi šablon, tedy
formátu těchto faktů. Následuje definice skupiny faktů (jméno \verb|world| této
skupiny je pouze informativní). Tyto fakty jsou přidány do znalostní báze, ze
které je po volání \verb|(reset)| inicializována pracovní paměť.

Pracovní paměť může být dále v průběhu inference modifikována třemi makry:
\begin{itemize}
  \item \verb|assert| přidává fakt(a) do pracovní paměti,
  \item \verb|retract| fakt(a) z pracovní paměti odebírá a
  \item \verb|modify| přímo modifikuje existující fakta.
\end{itemize}

Poté následuje definice tří odvozovacích pravidel - \verb|move|, \verb|push|
a~\verb|stop|. Definice každého pravidla sestává z množiny podmíněk, tedy
předpokladů pro jeho splnění (a následnou aktivaci), a množiny důsledků, tedy
libovolných lispových výrazů, které jsou při aktivaci pravidla vyhodnoceny.
Tyto dvě množiny jsou od sebe odděleny symbolem \verb|=>|. Podmínky pravidel
jsou ve formě vzorů (\emph{pattern}). Ty jsou velmi podobné faktům, ale mohou se
v nich vyskytovat proměnné (atomy začínající symbolem otazníku).
TODO: splnění podmínky pravidla
Krom toho mohou
být podmínky negovány, tedy podmínka je splněna tehdy, pokud neexistuje žádný
fakt, který by jí odpovídal.

\FloatBarrier

%%%%%%%%%%%%%%%%%%%%%%%%%%%%%%%%%%%%%%%%%%%%%%%%%%%%%%%%%%%%%%%%%%%%%%%%%%%%%%%%
\subsubsection{Definice znalostní báze}
\begin{framed}
  \begin{itemize}
    \item simple/templated fakta $\rightarrow$ templaty
    \item skupiny faktů
    \item pravidla
      \begin{itemize}
        \item podmínky podrobně - patterny, konjunkce, negace
        \item důsledky - typicky manipulace pracovní paměti
      \end{itemize}
    \item queries - fact-groups, rules + finders
  \end{itemize}
\end{framed}

ExiL, stejně jako CLIPS, rozlišuje dva typy faktů - jednoduché (\emph{simple,
ordered}) a strukturované (\emph{templated}). Stuktura jednoduchého faktu je udána
pouze pořadím atomů, typickou volbou je např. \verb|objekt-attribut-hodnota|:
\cl|(box color red),| či \verb|relace-<zůčastněné objekty>|: \cl|(in box hall).|
Strukturované fakty mají naproti tomu explicitně pojmenované položky (sloty),
typicky tedy popisují objekt s množinou atributů: \cl|(box :color red :size small),|
či relaci s jasně danými aktory: \cl|(in :object box :location hall),| kde
\verb|box| a \verb|in| jsou šablony, které je třeba definovat předem, jak záhy
uvidíme. Na pořadí specifikace slotů u strukturovaných faktů pochopitelně
nezáleží. Vyjadřovací síla obou skupin faktů je samozřejmě stejná, použitím
explicitnějších strukturovaných faktů, ale docílíme lepší čitelnosti a
jednoznačnější sémantiky exilového programu, zláště např. v případě relací na
jedné množině: \cl|(father john george).|

%%%%%%%%%%%%%%%%%%%%%%%%%%%%%%%%%%%%%%%%%%%%%%%%%%%%%%%%%%%%%%%%%%%%%%%%%%%%%%%%
\subsubsection{Manipulace pracovní paměti}
\begin{framed}
  \begin{itemize}
    \item inicializace ze znalostní báze
    \item ruční manipulace - assert, retract, modify
    \item queries - facts
  \end{itemize}
\end{framed}

%%%%%%%%%%%%%%%%%%%%%%%%%%%%%%%%%%%%%%%%%%%%%%%%%%%%%%%%%%%%%%%%%%%%%%%%%%%%%%%%
\subsubsection{Inference}
\begin{framed}
  \begin{itemize}
    \item fáze podrobně
    \begin{itemize}
      \item vyhodnocení podmínek - vazby (speciální - singleton, navázání faktu),
        konzistence
      \item výběr pravidla - strategie
      \item aktivace - vyhodnocení důsledků (typicky manipulace pracovní paměti)
        navázání proměnných, eval
    \end{itemize}
    \item spuštění inference, krokování (může se prolínat s ručnímanipulací w.m.)
    \item queries - agenda, strategies
  \end{itemize}
\end{framed}

%%%%%%%%%%%%%%%%%%%%%%%%%%%%%%%%%%%%%%%%%%%%%%%%%%%%%%%%%%%%%%%%%%%%%%%%%%%%%%%%
\subsubsection{Reset prostředí}
\begin{framed}
  \begin{itemize}
    \item durable/volatile slots
    \item clean, reset, complete reset (neměl by se jmenovat complete clean?)
  \end{itemize}
\end{framed}

%%%%%%%%%%%%%%%%%%%%%%%%%%%%%%%%%%%%%%%%%%%%%%%%%%%%%%%%%%%%%%%%%%%%%%%%%%%%%%%%
\subsubsection{Sledování průběhu inference}
\begin{framed}
  \begin{itemize}
    \item watchery
  \end{itemize}
\end{framed}

%%%%%%%%%%%%%%%%%%%%%%%%%%%%%%%%%%%%%%%%%%%%%%%%%%%%%%%%%%%%%%%%%%%%%%%%%%%%%%%%
\subsubsection{Undo/redo}
\begin{framed}
  \begin{itemize}
    \item lze použít na všechny funkce/makra s vedlejším efektem
    \item pokud funkce nemá vedlejší efekt (fakt neexistuje, apod.), nezapíše se
      undo step
    \item queries - undo-stack, redo-stack
  \end{itemize}
\end{framed}

%%%%%%%%%%%%%%%%%%%%%%%%%%%%%%%%%%%%%%%%%%%%%%%%%%%%%%%%%%%%%%%%%%%%%%%%%%%%%%%%
\subsubsection{Zpětné řetězení}
\begin{framed}
  \begin{itemize}
    \item cíle jako patterny
    \item základní inference - nejdřív fakta, pak pravidla, v jakém pořadí
      vybírá
    \item alternativní odpovědi - backtracking
  \end{itemize}
\end{framed}

%%%%%%%%%%%%%%%%%%%%%%%%%%%%%%%%%%%%%%%%%%%%%%%%%%%%%%%%%%%%%%%%%%%%%%%%%%%%%%%%
\subsubsection{Práce s více prostředími}

%%%%%%%%%%%%%%%%%%%%%%%%%%%%%%%%%%%%%%%%%%%%%%%%%%%%%%%%%%%%%%%%%%%%%%%%%%%%%%%%
\subsubsection{Grafické uživatelské rozhraní}
