%%%%%%%%%%%%%%%%%%%%%%%%%%%%%%%%%%%%%%%%%%%%%%%%%%%%%%%%%%%%%%%%%%%%%%%%%%%%%%%%
\subsection{Uživatelská příručka}
\subsubsection{Základní pojmy}
Nyní stručně zadefinuji základní pojmy, nutné pro pochopení fungování knihovny
ExiL a práci s ní. K těmto pojmům se posléze v teoretické části textu vrátím
a jejich popis rozšířím o další souvislosti.
\begin{description}[leftmargin=5cm,style=sameline]
  \item[problémová doména] množina pojmů relevantních pro řešení určité skupiny
    problémů
  \item[fakt] elementární statická znalost - tvrzení
  \item[(odvozovací) pravidlo] elementární odvozovací znalost - pokud víme, že
    (ne)platí nějaká tvrzení, můžeme odvodit, že platí i~nějaká další
  \item[znalost (v ExiLu)] množina faktů a pravidel
  \item[znalostní baze] výchozí znalost
  \item[working memory] aktuální množina faktů
  \item[production memory] aktuální množina pravidel
  \item[inference] odvozování - postupná aplikace odvozovacích pravidel
    (s případnými zásahy uživatele)
    (+ výpočet splněných pravidel, jejich výběr)
\end{description}
Pojmy working memory a production memory lze samozřejmě doslova přeložit, ale
žádný z překladů mi nepřijde příliš názorný (zvlášť vzhledem k tomu, že ve
skutečnosti nejde o paměť, nýbrž aktuální obsah pomyslné paměti), budu tedy
používat anglická spojení užívaná v literatuře. Představa pojmů bude jasnější,
jakmile si je ukážeme na příkladech.

TODO: dořešit, zda citovat zde, nebo až v teorii

(systém ve skutečnosti pracuje s \emph{reprezentací} znalostí, ale to nás zatím
nezajímá)

ExiL, stejně jako CLIPS, rozlišuje dva typy faktů - jednoduché (\emph{simple,
ordered}) a strukturované (\emph{templated}). Stuktura jednoduchého faktu je udána
pouze pořadím atomů, typickou volbou je např. \verb|objekt-attribut-hodnota|:
\cl|(box color red),| či \verb|relace-<zůčastněné objekty>|: \cl|(in box hall).|
Strukturované fakty mají naproti tomu explicitně pojmenované položky (sloty),
typicky tedy popisují objekt s množinou atributů: \cl|(box :color red :size small),|
či relaci s jasně danými aktory: \cl|(in :object box :location hall),| kde
\verb|box| a \verb|in| jsou šablony, které je třeba definovat předem, jak záhy
uvidíme. Na pořadí specifikace slotů u strukturovaných faktů pochopitelně
nezáleží. Vyjadřovací síla obou skupin faktů je samozřejmě stejná, použitím
explicitnějších strukturovaných faktů, ale docílíme lepší čitelnosti a
jednoznačnější sémantiky exilového programu, zláště např. v případě relací na
jedné množině: \cl|(father john george).|

%%%%%%%%%%%%%%%%%%%%%%%%%%%%%%%%%%%%%%%%%%%%%%%%%%%%%%%%%%%%%%%%%%%%%%%%%%%%%%%%
\subsubsection{Struktura programu}

% Popsat jednotlivé sekce kódu, jejich význam (korespondence s fázemi návrhu ES,
% formulace problému, formát dat, vstupní znalosti, odvozovací krok, řízení
% odvozování, ladění)
% \begin{itemize}
%   \item definice prostředí
%   \item definice šablon - formát dat (simple, template)
%   \item definice znalostní báze - vstupní znalost - deffacts, defrules
%   \item (nastavení sledování průběhu inference - watchers)
%   \item (úprava průběhu inference - strategie)
%   \item spuštění / krokování inference - reset, run, step
%   \item dotazy nad working memory - facts, agenda
%   \item úprava working memory - assert, retract, modify
%   \item dotazy nad znalostní bází - fact-groups, rules
%   \item cleanup - volatile vs durable sloty prostředí
%   \item undo/redo
%   \item zpětné řetězení
%   \item GUI
% \end{itemize}

% Ke každému ze základních pojmů uvést příklad z programu.

Příklad \ref{typical structure} (na straně \pageref{typical structure}) ukazuje
typickou strukturu programu nad knihovnou ExiL (dále exilový program). Tento
program používá strukturovaných faktů, začíná tudíž definicemi šablon, tedy
formátu těchto faktů. Následuje definice skupiny faktů (jméno \verb|world| této
skupiny je pouze informativní). Tyto fakty jsou přidány do znalostní baze, ze
které je po volání \verb|(reset)| inicializována \emph{working memory}.

\begin{listing}[H]
\caption{Základní struktura exilového programu}
\label{typical structure}
\begin{clcode}
;; knowledge structure
(deftemplate goal action object from to)
(deftemplate in object location)

;; initial knowledge
(deffacts world
  (in :object robot :location A)
  (in :object box :location B)
  (goal :action push :object box :from B :to A))

;; inference rules
(defrule move
  (goal :action push :object ?obj :from ?from)
  (in :object ?obj :location ?from)
  (- in :object robot :location ?from)
  ?robot <- (in :object robot :location ?)
  =>
  (modify ?robot :location ?from))

(defrule push
  (goal :action push :object ?obj :from ?from :to ?to)
  ?object <- (in :object ?obj :location ?from)
  ?robot <- (in :object robot :location ?from)
  =>
  (modify ?robot :location ?to)
  (modify ?object :location ?to))

(defrule stop
  ?goal <- (goal :action push :object ?obj :to ?to)
  (in :object ?obj :location ?to)
  =>
  (retract ?goal)
  (halt))

;; initiation
(reset)

;; inference execution
; (step)
(run)
\end{clcode}
\end{listing}

Working memory může být dále v průběhu inference modifikována třemi makry:
\begin{itemize}
  \item \verb|assert| přidává fakt(a) do working memory,
  \item \verb|retract| fakt(a) z working memory odebírá a
  \item \verb|modify| přímo modifikuje existující fakta.
\end{itemize}

Poté následuje definice tří odvozovacích pravidel - \verb|move|, \verb|push|
a~\verb|stop|. Definice každého pravidla sestává z množiny podmíněk, tedy
předpokladů pro jeho splnění (a následnou aktivaci), a množiny důsledků, tedy
libovolných lispových výrazů, které jsou při aktivaci pravidla vyhodnoceny.
Tyto dvě množiny jsou od sebe odděleny symbolem \verb|=>|. Podmínky pravidel
jsou ve formě vzorů (\emph{pattern}). Ty jsou velmi podobné faktům, ale mohou se
v nich vyskytovat proměnné (atomy začínající symbolem otazníku).
TODO: splnění podmínky pravidla
Krom toho mohou
být podmínky negovány, tedy podmínka je splněna tehdy, pokud neexistuje žádný
fakt, který by jí odpovídal.

\subsubsection{Definice znalostní báze}

\subsubsection{Manipulace working memory}

\subsubsection{Inference}














