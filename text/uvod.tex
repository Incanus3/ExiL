%%%%%%%%%%%%%%%%%%%%%%%%%%%%%%%%%%%%%%%%%%%%%%%%%%%%%%%%%%%%%%%%%%%%%%%%%%%%%%%%
\section{Úvod}

Pojem expertního systému spadá do oblasti umělé inteligence. Jde o počítačový
systém, který simuluje rozhodování experta nad zvolenou problémovou doménou.
Expertní systém může experta zcela nahradit, nebo mu při rozhodování asistovat.

Ve své bakalářské práci \cite{bakalarka} jsem implementoval základní knihovnu pro tvorbu
expertních systémů (tzv. prázdný expertní systém) s~dopředným řetězením v~jazyce
Common Lisp. Cílem této práce je knihovnu rozšířit o následující:
\begin{itemize}
  \item syntaktický režim pro zajištění přiměřené kompatibility se systémem
    CLIPS,
  \item možnost vrácení provedených změn včetně odvozovacích kroků,
  \item podpora pro ladění s jednoduchým grafickým uživatelským rozhraním pro
    prostředí LispWorks\texttrademark,
  \item rozšíření odvozovacího aparátu o základní zpětné řetězení.
\end{itemize}

Jazyk Common Lisp\footnote{\url{http://en.wikipedia.org/wiki/Common\_Lisp}}
(případně jiné dialekty Lispu) je častou volbou pro implementaci umělé
inteligence díky svým schopnostem v oblasti symbolických výpočtů (manipulace
symbolických výrazů), na nichž řešení těchto problémů často staví. Navíc jde
o~velmi vysokoúrovňový, dynamicky typovaný jazyk, díky čemuž je programový kód
expresivní, snadno pochopitelný a tudíž jednoduše rozšiřitelný.

Synax systému CLIPS\footnote{\url{http://clipsrules.sourceforge.net}} byla
zvolena proto, že jde o reálně používaný
systém\footnote{\url{http://clipsrules.sourceforge.net/FAQ.html\#Q6}}, jehož
syntax je Lispu velmi blízká, takže není těžké ji v~Lispu napodobit.

Přestože běžnou praxí je začínat diplomovou práci teoretickou částí, definovat
jednotlivé pojmy a principy a ty poté v praktické části uplatnit, rozhodl jsem
se postupovat opačně, tedy začít práci praktickou částí. Domnívám se totiž (také
na základě zkušeností nabytých při vypracování bakalářské práce), že je podstatně
snazší, minimálně v řešené problematice, pochopit příklady bez detailní znalosti
teorie, než snažit se pochopit teorii bez příkladů, na nichž si lze popisované
pojmy a principy představit. V praktické části tedy uvedu jen minimální množtví
teorie nutné pro pochopení práce s knihovnou, načež se k ní v teoretické části
textu vrátím, pojmy zadefinuji přesně a rozšířím o souvislosti. V tuto chvíli už
si bude čtenář schopen představit, jaké problémy lze pomocí expertního systému
řešit a jak takový expertní systém v praxi vypadá.

\begin{framed}
  Pro popsaní některé teorie (symbolické výpočty, ...) je navíc užitečná (ne-li
  potřebná) představa, jak expertní systém funguje uvnitř.
\end{framed}

\begin{framed}
  Kdo už to dělal, čím se tohle liší
\end{framed}
