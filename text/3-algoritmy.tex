%%%%%%%%%%%%%%%%%%%%%%%%%%%%%%%%%%%%%%%%%%%%%%%%%%%%%%%%%%%%%%%%%%%%%%%%%%%%%%%%
\section{Použité algoritmy}
Přiložená implementace expertního systému zahrnuje několik z~literatury
čerpaných algoritmů. V~prvé řadě je to algoritmus Rete, který se stará o~pattern
matching podmínek odvozovacích pravidel proti faktům znalostní baze, tedy o
zjišťování, která odvozovací pravidla je možno aplikovat v dalším kole výpočtu.
Poté je použito několik velice jednoduchých algoritmů pro výběr pravidla, jenž
bude jako další aktivováno.

\subsection{Algoritmus Rete}
Algoritmus Rete kontroluje splnění podmínek inferenčních pravidel fakty
znalostní baze. Použití naivního přístupu, tedy zkoušení platnosti všech
podmínek proti všem faktům při každé změně znalostní baze není možné, neboť
tato kontrola se již při několika stovkách faktů či desítkách pravidel
(závisí samozřejmě na výkonu výpočetního hardware, jeho zatížení, atd.)
stává příliš časově náročnou.

Algoritmus pracuje na základě dataflow sítě (či chcete-li sítě toku dat),
kterou průběžne konstruuje a upravuje při každém přidání či odebrání
inferenčního pravidla. Přepočet splněných pravidel algoritmem Rete je o tolik
rychlejší díky tomu, že tato síť uchovává v každém uzlu výsledky z minulých
výpočtů a nově přidaný (či odebraný) fakt touto sítí jen velice rychle
\uv{proteče} a aktualizuje výsledky jen ve velmi malé části uzlů.

Dataflow síť algoritmu je z pohledu teorie grafů spojitý orientovaný graf.
Máme-li dva uzly a orientovanou hranu, která je spojuje, budu uzel, z nějž
hrana vede, označovat jako rodiče a uzel, do nějž vede, jako potomka.
Síť je tvořena dvěma částmi označovanými jako alfa a beta.

Alfa část sítě řeší konstantní porovnávání přidávaných faktů s patterny
v podmínkách pravidel, ignoruje tedy vazby proměnných a jejich konzistenci.
Jde v podstatě o strom, jehož každý uzel testuje právě jeden atom faktu.
Má-li atom požadovanou hodnotu, předá uzel fakt svým potomkům a ty pokračují
v testování.
