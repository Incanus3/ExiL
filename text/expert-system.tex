\subsection{Expertní systémy}

\begin{framed}
  \begin{itemize}
    \item co je expertní systém - vlastnosti, distinkce
    \item popsáno v bakalářce:
      \begin{itemize}
        \item základní definice
        \item historie
        \item zařazení (knowledge-based, rule-based, AI)
        \item postup návrhu (volba reprezentace znalostí, proces učení -
          vytvoření znalostní báze)
        \item typické vlastnosti (dedukce, ne algebraické výpočty, oddělení
          znalosti od odvozovacího aparátu, heuristiky)
        \item základní rozdíl mezi dopředným a zpětným řetězením
      \end{itemize}

    \item zopakovat definici - počítačový program, který usuzuje nad nějakou
      problémovou doménou za účelem vyřešení zadaného problému (např. nalézt
      posloupnost akcí), nebo poskytnutí odpovědi na otázku (zda odpověď
      existuje, nalézt možné vazby proměnných)
    \item problémová doména, vyřešení problému, odpověď na otázku

    \item vyjmenovat, co už řeší bakalářka
    \item typickou vlastností je také schopnost vysvětlit, jak se došlo k
      řešení (to zde řeší watchers, zpětná inference řeší přímo) - motivace
      (str. 9)
    \item typické problémy řešené pomocí ES, aplikabilita
      \begin{itemize}
        \item plánování posloupnosti akcí
        \item interpretace dat (např. z nějakých senzorů), strukturální analýza
          (např. aplikace chemických analytických metod - když vypadá
          chromatograf takhle, pak látka asi obsahuje tohle)
        \item diagnóza poruch systému
        \item hledání konfugurace komplexních objektů - např. provisioning
          serverů pro pokrytí zadaného úkolu
      \end{itemize}
    \item aplikabilita, limitace
  \end{itemize}
\end{framed}

Expertní systém je počítačový program, který usuzuje nad nějakou problémovou
doménou za účelem vyřešení zadaného problému (např. nalézt posloupnost akcí),
nebo poskytnutí odpovědi na otázku (zda odpověď existuje, nalézt možné vazby
proměnných)
