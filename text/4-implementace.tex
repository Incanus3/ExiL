%%%%%%%%%%%%%%%%%%%%%%%%%%%%%%%%%%%%%%%%%%%%%%%%%%%%%%%%%%%%%%%%%%%%%%%%%%%%%%%%
\section{Popis implementace}
Knihovna je napsána v programovacím jazyku Common Lisp. K jejímu vývoji jsem
použil textový editor \textsf{GNU Emacs} s rozšířením \textsf{SLIME} (The
Superior Lisp Interaction Mode for Emacs) a kompilátor \textsf{SBCL} (Steel Bank
Common Lisp). Vývoj jsem prováděl na architektuře \textsf{GNU/Linux x86}.
Funkčnost knihovny jsem testoval také ve vývojovém prostředí
LispWorks\textsuperscript{\textregistered}.

Zdrojový kód knihovny je rozdělen celkem do 15 soborů:
\begin{verbatim}
        packages.lisp
        utils.lisp
        templates.lisp
        facts.lisp
        patterns.lisp
        rules.lisp
        rete-generic-node.lisp
        rete-alpha-part.lisp
        rete-beta-part.lisp
        rete-net-creation.lisp
        matches.lisp
        activations.lisp
        strategies.lisp
        environment.lisp
        export.lisp
\end{verbatim}
\begin{itemize}
\item V souboru \verb|packages.lisp| definuji package (\uv{balíček}) \verb|exil|,
který bude definované funkce, metody a makra obsahovat.
\item Soubor \verb|utils.lisp| obsahuje jednoduché obecné funkce a makra, které
jsem během vývoje pro zjednodušení další práce napsal.
\item V souboru \verb|export.lisp| jsou všechny metody a makra, které knihovna
uživateli poskytuje. Tyto jsou z package \verb|exil| exportovány, lze je
tedy volat bez nutnosti vstupu do package (před název je třeba připojit
``\verb|exil:|'').
\item Ve zbytku souborů definuji jednotlivé třídy a metdody. Kód v každém ze
souborů staví na konstruktech definovaných v předchozích souborech. Prostředí
(třída \verb|environment|), definovaná ve stejnojmenném souboru nakonec všechny
dříve definované třídy zakomponuje do jednoho celku a vytvoří celek, na kterém
může soubor \verb|export| vystavět makra, která uživatel bude nakonec používat.
\end{itemize}
%%%%%%%%%%%%%%%%%%%%%%%%%%%%%%%%%%%%%%%%%%%%%%%%%%%%%%%%%%%%%%%%%%%%%%%%%%%%%%%%
\subsection{Templates}
Soubor vytváří základní mechanismus pro práci se strukturovanými daty.
Třída \verb|template| obsahuje sloty pro uchování informací o uživatelsky
definovaných šablonách. Všechna strukturovaná data nebo patterny jsou pak
odvozenými instancemi třídy \verb|template-object|. Při vytváření těchto
instancí mechanismus prochází kromě inicializačních parametrů také specifikaci
slotů uloženou v konkrétní instanci třídy \verb|template| a tak umožňuje
nastavení implicitnch hodnot. Krom toho soubor definuje predikát
\verb|tmpl-object-specification-p|, který podle specifikace faktu (resp.
patternu) určí, jde-li o fakt jednoduchý či strukturovaný.
%%%%%%%%%%%%%%%%%%%%%%%%%%%%%%%%%%%%%%%%%%%%%%%%%%%%%%%%%%%%%%%%%%%%%%%%%%%%%%%%
\subsection{Facts a Patterns}
