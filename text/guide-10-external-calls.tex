%%%%%%%%%%%%%%%%%%%%%%%%%%%%%%%%%%%%%%%%%%%%%%%%%%%%%%%%%%%%%%%%%%%%%%%%%%%%%%%%
\subsubsection{Volání ExiLu z jiného kódu a naopak}

Doposud jsem v ukázkách práce s ExiLem pracoval většinou s makry. Byla to
následující:
\begin{description}[leftmargin=6.7cm,style=sameline,align=right,labelsep=0.5cm]
  \item[definice šablon] \verb|deftemplate undeftemplate| \verb|find-template|
  \item[definice skupin faktů] \verb|deffacts undeffacts find-fact-group|
  \item[definice pravidel] \verb|defrule undefrule find-rule|
  \item[modifikace pracovní paměti] \verb|assert retract modify|
  \item[definice cílů] \verb|defgoal undefgoal|
  \item[sledování průběhu inference] \verb|watch unwatch watchedp|
  \item[strategie výběru shody] \verb|defstrategy setstrategy|
  \item[definice prostředí] \verb|defenv undefenv setenv|
\end{description}
Tato makra berou jako parametry symboly a/nebo seznamy a tyto automaticky
\emph{quotují}. To je pohodlné, pracujeme-li s knihovnou přímo. Představme si
ale, že chceme knihovnu volat z jiného kódu a například specifikace faktů, které
předáváme makru \verb|assert|, generovat nějakou funkcí.

Protože makro \verb|assert| seznamy se specifikacemi faktů quotuje, místo aby je
vyhodnotilo, nelze ho v tomto případě použít. Výsledkem volání
\cl|(assert (generate-fact))| totiž bude přidání faktu \verb|(generate-fact)| do
pracovní paměti. K tomuto účelu poskytuje ExiL ke všem uvedeným makrům funkční
alternativy, které parametry vyhodnocují. Tyto jsou označeny suffixem \verb|f|,
například \verb|assertf|. Tímto suffixem sice v Lispu typicky označujeme
destruktivní makra, která mění svůj argument, v případě exilových maker ale
záměna nehrozí.

Dotazovací funkce a makra jako \verb|facts|, \verb|goals|,
\verb|find-fact-group|, apod. navíc nevypisují hodnoty na výstup, nýbrž vrací
externí reprezentaci objektů, se kterou je možné dále manipulovat a pak ji třeba
systému předat zpátky. To umožnuje například následující (nepříliš užitečné)
volání:
\begin{minted}{cl}
(dolist (fact (facts))
  (retractf fact)
  (assertf (cons 'my-fact fact))).
\end{minted}

Funkční alternativy jako \verb|assertf| také umožnují použití složitějších
konstruktů v důsledcích pravidla. Bude-li například podmínka pravidla
\begin{minted}{cl}
(defrule surround-by-as
  (palindrome ?p)
  =>
  (assertf `(palindrome ,(concatenate 'string "a" ?p "a"))))
\end{minted}
splněna faktem \verb|(palindrome "b")|, přibyde po jeho aktivaci do pracovní
paměti fakt \verb|(palindrome "aba")|. Funkci \verb|assertf| bohužel nelze
použít se zpětnou inferencí, neboť ta nemá šanci předvídat, jaký fakt bude
jejím voláním do pracovní paměti přidán.

V důsledcích pravidla bychom také mohli chtít například upozornit jinou část
programu na událost, ke které došlo. Protože ExiL nahrazuje výskyty
proměnných v celé důsledkové části pravidla, lze toho snadno dosáhnout. Je ale
třeba dát pozor na quotování hodnot proměnných. Uvažme pravidlo
\begin{minted}{cl}
(defrule move-robot
  (goal move robot ?from ?to)
  (in robot ?from)
  =>
  (retract (in robot ?from))
  (assert (in robot ?to))
  (notify 'moving-robot '?from '?to))
\end{minted}
a fakt \verb|(goal move robot A B)| v pracovní paměti. Kdybychom ve volání
\verb|notify| proměnné \verb|?from| a \verb|?to| nequotovali, volání by se při
aktivaci pravidla vyhodnotilo jako \verb|(notify 'moving-robot A B)|. To by
pravděpodobně skončilo chybovou hláškou sdělující, že proměnné \verb|A| a
\verb|B| nejsou definovány.

Makra \verb|deftemplate|, \verb|deffacts|, \verb|defrule| a \verb|modify| berou
specifikace slotů šablony, faktů, těla pravidla a změn k provedení (v tomto
pořadí) jako další parametry. Díky tomu nemusíme tyto parametry obalovat do
dalšího seznamu. Jejich funkční alternativy naproti tomu očekávají tyto
parametry v jednom seznamu, například
\cl|(deffactsf 'world (list '(in box A) '(in robot B))).|
To umožňuje snazší generování těchto specifikací funkcemi.
