\subsection{Reprezentace znalostí}
\label{knowledge representation}

% representation language - oriented towards organizing descriptions
% of objects and ideas, rather than stating sequences of instructions or
% storing simple data elements

Aby bylo možné pracovat se znalostmi v expertním systému, musí tyto splňovat
několik vlastností:
\begin{itemize}
  \item přesnost - pro vyvození rozumných závěrů nemůže být znalost vágní (to
    neznamená, že nemůžeme vyjádřit nejistotu, i ta ale musí být definována
    přesně),
  \item relevance - znalosti by se měly týkat pouze pojmů z problémové domény
  \item organizovanost - znalosti by měly postihovat všechny potřebné spojitosti
    mezi pojmy (to zahrnuje například používání vždy stejného pojmu pro tutéž
    věc),
  \item uzavřenost - systém by pro vyvození závěrů neměl potřebovat žádné
    předchozí znalosti.
\end{itemize}

Systém nechápe význam znalostí, pracuje pouze s jejich reprezentací.
Reprezentaci obecně chápeme jako množinu syntaktických a sémantických pravidel,
která umožňují popsat nějaké entity. Reprezentaci entity pak jako výraz, který
podle těchto pravidel popisuje danou entitu. Jazyk reprezentace je množina
všech možných reprezentací entit. V expertních systémech jsou těmito entitami
právě znalosti.

Syntax reprezentace je množina pravidel, která definují základní stavební bloky
(atomy) jazyka reprezentace a jak jejich kombinací vytvářet výrazy tohoto
jazyka, tedy reprezentace jednotlivých entit. Sémantika říká, jak jsou výrazy
jazyka reprezentace interpretovány - jaký je jejich význam. Sémantika nám tedy
dává zobrazení z reprezentace entity na reprezentovanou entitu. Sémantická
pravidla jsou často rekurentní - přiřkneme význam atomům a poté jednotlivým
strukturám, které vznikají jejich kombinací.

Aby byla reprezentace praktická, musí být
\begin{itemize}
  \item jednoznačná - není možné, aby měl jeden výraz jazyka reprezentace
    několik významů,
  \item expresivní - reprezentace musí umožňovat vyjádření všech potřebných
    detailů (\emph{distinction}) popisované entity,
  \item srozumitelná - význam výrazu by měl být snadno pochopitelný bez
    nutnosti chápat, jak je systémem zpracováván,
  \item stručná - výrazy by mělo být snadné psát, neměly by být zbytečně
    mnohomluvné;
  \item reprezentaci musí být možno zpracovávat a ukládat v počítači.
\end{itemize}
Míra expresivity je často v protikladu snadného strojového zpracování, hledáme
tedy kompromis mezi těmito dvěma vlastnostmi.

Často používanými reprezentačními schematy jsou strukturované objekty,
odvozovací pravidla a logické programy. Knihovna ExiL, stejně jako systém CLIPS,
používá prvních dvou. Typickou ukázkou třetího je jazyk
Prolog\footnote{\url{http://en.wikipedia.org/wiki/Prolog}}.

Expertní systémy jsou \emph{systémy založené na
pravidlech}\footnote{\url{http://en.wikipedia.org/wiki/Rule-based\_system}}.
Jako takové rozlišují dva základní typy znalostí - \emph{fakty} a
\emph{pravidla}. Fakt je elementární \uv{statická} znalost - vyjadřuje nějaké
tvrzení z problémové domény, např. \uv{obloha je jasná}. Odvozovací pravidlo je
pak elementární odvozovací znalostí a má formu implikace. Pravidlo vyjadřuje,
jaké (jaká) tvrzení můžeme odvodit z platnosti jiných tvrzení, např. \uv{pokud
je obloha jasná, neprší}. Pravidlo tedy sestává z \emph{podmínek} (\uv{obloha je
jasná}) a \emph{důsledků} (\uv{neprší}).

\emph{Stav systému} je tvořen množinou aktuálně platných tvrzení a reprezentován
množinou reprezentací odpovídajících faktů. Tvrzení tedy platí, pokud je
reprezentace příslušného faktu součástí stavu systému. Počáteční stav
expertního systému spolu s množinou definovaných odvozovacích pravidel tvoří
tzv. \emph{znalostní bázi} systému. Pokud se může množina definovaných pravidel
v průběhu práce se systémem měnit, zařazujeme ji také do stavu systému. Ten je
pak reprezentován dvěma množinami - množinou reprezentací faktů (tzv.
\emph{pracovní paměť}) a~množinou definovaných pravidel (tzv. \emph{produkční
paměť}).

Pojmy pracovní a produkční paměť nejsou příliš intuitivní. Jde o doslovný
překlad v~literatuře užívaných pojmů \emph{working memory} a \emph{production
memory}. Pojem production memory je odvozen od označení \emph{production
rules}\footnote{\url{http://en.wikipedia.org/wiki/Production\_system}} pro
odvozovací pravidla. Nejde ve skutečnosti o paměti, nýbrž o obsahy pomyslných
pamětí. Pojmy \uv{pracovní množina faktů} a \uv{pracovní množina pravidel} by
byly jistě výstižnější, bohužel ale také značně těžkopádné.

Různé expertní systémy se liší v reprezentaci faktů a pravidel. V různých
expertních systémech tedy můžeme vyjádřit znalosti s různou mírou flexibility.
Jednoduchým příkladem reprezentace faktů a pravidel je reprezentace prázného
expertního systému STRIPS\footnote{\url{http://en.wikipedia.org/wiki/STRIPS}}.
Atomy syntaxe této reprezentace jsou symboly. Ty reprezentují názvy jednotlivých
objektů a relací z problémové domény. Fakty jsou pak ve tvaru
\verb|<relace>(<argumenty oddělené čárkami>)|, např. \verb|At(robot,roomA)|.

Pro vyšší flexibilitu pravidel definuje často syntaxe expertního systému
speciální výraz, tzv. \emph{vzor}. Ten má typicky stejnou strukturu jako fakt,
ale může obsahovat speciální atomy - \emph{proměnné}. Podmínky a důsledky
pravidel jsou pak tvořeny nikoli fakty, nýbrž právě vzory. Díky tomu mohou být
podmínky pravidla splněny mnoha různými posloupnostmi faktů.

Atomy proměnných začínají v~reprezentaci STRIPSu velkými písmeny. STRIPSovým
vzorem je tedy například \verb|At(Box,Room)|. STRIPS umožňuje použití proměnných
jen pro argumenty relace, samotný symbol názvu relace je neproměnný, navíc už
velkým písmenem začíná (což je poněkud matoucí).

Pro potřeby vyhodnocování podmínek odvozovacích pravidel zavedu relaci
\emph{kongruence} faktu se vzorem (příp. faktu a vzoru, nebo i naopak). Fakt
a~vzor jsou kongruentní, pokud mají stejnou strukturu, tedy stejný název
predikátu a stejný počet jeho argumentů, a pokud jsou neproměnné atomy vzoru
stejné jako atomy na odpovídajících pozicích faktu. Vzor \verb|At(Box,roomA)|
je tedy kongruentní s faktem \verb|At(box1,roomA)|, nikoli však s fakty
\verb|At(box1,roomB)|, \verb|At(box1,roomA,floor2)| a \verb|In(box1,roomA)|.
Posloupnost faktů je kongruentní s posloupností vzorů, pokud mají tyto stejnou
délku a jednotlivé fakty jsou kongruentní s odpovídajícími vzory (v daném
pořadí).

Podmínky STRIPSového odvozovacího pravidla mohou být definovány například takto:
\begin{verbatim}
Preconditions: At(Box,Location), At(lift,Location), Level(Box,low).
\end{verbatim}
Jednotlivé podmínky pravidla jsou sémanticky spojeny logickou konjunkcí a celá
posloupnost je existenčně kvantifikována. Při vyhodnocování podmínek pravidla
tedy hledáme (alespoň jednu) posloupnost faktů kongruentní s posloupností vzorů
podmínek. Každý vzor podmínky je tedy spárován s jedním kongruentním faktem.

Konkrétní atom, který se ve faktu vyskytuje na pozici, kde je ve vzoru
odpovídající podmínky proměnná, označujeme jako \emph{vazbu} této proměnné. Při
vyhodnocování podmínek je zajištěna konzistence těchto vazeb. Vyskytuje-li se
táž proměnná na více místech ve vzorech podmínek pravidla, musí mít vždy stejnou
vazbu. Pokud tedy najdeme posloupnost faktů kongruentní s podmínkami pravidla
při zachování konzistence vazeb, říkáme, že podmínky pravidla jsou splněny touto
posloupností faktů.

Předchozí podmínky jsou splněny například posloupností faktů \verb|At(box1,z)|,
\verb|At(lift,z)| a \verb|Level(box1,low)|. Použité vazby proměnných jsou
\verb|box1| pro \verb|Box| a~\verb|z|~pro \verb|Location|. Kdyby ovšem byla v
druhém faktu pozice \verb|x|, nesplňovala by tato posloupnost uvedené podmínky,
neboť by nebyla zachována konzistence vazeb.

Kdyby byly podmínky spojeny logickou disjunkcí, stačilo by nám nalézt jeden
fakt kongruentní alespoň s jednou z podmínek. Kdyby byly podmínky naopak
kvantifikovány všeobecně, znamenalo by to, že všechny posloupnosti faktů
kongruentní s podmínkami pravidla musí zachovávat konzistenci vazeb proměnných.
Některé expertní systémy umožňují definovat i takovéto podmínky pravidel.

Odvozovací pravidlo systému STRIPS je reprezentováno názvem, seznamem použitých
proměnných, podmínkami a důsledky. Podmínky i důsledky pravidel jsou
reprezentovány posloupnostmi vzorů. Před vzorem v důsledcích pravidla může být
navíc použit speciální symbol \verb|not|. Definice odvozovacího pravidla STRIPSu
může tedy vypadat následovně \cite{strips}:
\begin{verbatim}
// move the box up a level
_MoveUp(Box,Location)_
Preconditions:  At(Box,Location), At(lift,Location), Level(Box,low)
Postconditions: Level(Box,high), not Level(Box,low).
\end{verbatim}

\emph{Aplikace odvozovacího pravidla} definuje přechod mezi dvěma stavy
systému. Pravidlo je možné aplikovat, jsou-li splněny jeho podmínky. Při jeho
aplikaci jsou vyhodnoceny jeho důsledky, na základě čehož je stav systému
modifikován - reprezentace nějakých faktů jsou přidány do a/nebo odebrány z
pracovní paměti systému.

Při aplikaci odvozovacího pravidla systému STRIPS jsou nejprve nahrazeny
proměnné vzorů důsledků pravidla jejich vazbami získanými při vyhodnocování
jeho podmínek. Tím převedeme vzory důsledků pravidla na fakty. Každý z těchto
faktů je pak přidán do pracovní paměti, pokud není před vzorem odpovídajícího
důsledku použit symbol \verb|not|, nebo z paměti odebrán, pokud je tento symbol
použit.

\begin{framed}
odněkud dále pracovat s pojmem znalost (reprezentace je implicitní)
\end{framed}
