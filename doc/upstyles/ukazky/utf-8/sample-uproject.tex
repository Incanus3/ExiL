%%%  Vzor pro použití makra pro ročníkovou práci, soubor byl revidován 
%%%  a doplněn v září 2001
%%%  (c) 2001 Vilém Vychodil, <vilem.vychodil@upol.cz>
%%%  Vzorový soubor upraven v květnu 2009
%%%  (c) 2009 Jan Outrata, <jan.outrata@upol.cz>
%%%
%%%  Po přeložení programem CSLaTeX (třikrát) je potřeba použít
%%%  program DVIPS a takto získaný PostScriptový soubor vytisknout
%%%  na PostScriptové tiskárně nebo pomocí programu GhostScript.
%%%
%%%  Rovněž je možné použít program DVIPDFM a vytvořit z dokumentu
%%%  soubor ve formátu PDF včetně hypertextových odkazů.


%%% Deklarace hlavičky dokumentu.
\documentclass{article}

%%% Připojení dodatečného stylu pro ročníkovou práci. Nepovinné
%%% argumenty `tables' a `figures' použijte pouze v případě, že váš
%%% dokument obsahuje tabulky a obrázky a chcete vytvořit jejich
%%% seznamy za obsahem.
%%%
%%% Argument `joinlists' způsobí zřetězení obsahu a seznamů tabulek a obrázků.
%%% Není-li použít, všechny seznamy jsou uvedeny na samostatných stránkách.
%%%
%%% Pokud chcete vytvářet pouze dokument ve formátu PostScript, můžete uvést
%%% dodatečný argument `nopdf'. Tím se potlačí chybová hlášení při použití
%%% programu `dvips'.
\usepackage[tables,figures]{uproject}

%%% Dodatečné standardní styly.
\usepackage[utf8]{inputenc}

%%% Parametry pro vytvoření úvodních stránek. Makrem \subtitle je možné
%%%  vytvořit druhý řádek v názvu projektu.
\title{Název projektu}
%\subtitle{Druhý řádek názvu}
\author{Jan Novák}
\group{Aplikovaná informatika, II. ročník}
\date{Květen 2009}

%%% Pomocí \docinfo je možné vytvořit název pro PDF dokument, zpravidla je
%%% dobré použít předcházející název, ale bez diakritiky. Možné je však zvolit
%%% úpolně jiný výstižný název. Při tvorbě PostScriptu bude příkaz ignorován.
\docinfo{Jan Novak}{Nazev projektu}

%%% Zadání abstraktu. Pouze jeden odstavec!
\abstract{%
Tento projekt řeší úkoly, které byly určeny zadáním projektu.
Tato anotace by měla stručně popisovat zpracovaný projekt a neměla by
přitom přesáhnout zhruba 10 řádků. V~žádném případě by neměla být rozdělena
do více odstavců.}

\begin{document}

%%% Vytvoření úvodních stránek, obsahu a seznamu tabulek a obrázků.
\maketitle


%%% Popis a analýza.
\newpage
\section{Popis a analýza}
Toto je první část dokumentace a provedu zde rozbor zadání.

\subsection{První podkapitola}
Toto je moje první podkapitola.
Zde jsem vycházel z~prací kolegy Složitého. Lze jej kontaktovat 
na adrese \mail{tomas.slozity@tezke.projekty.cz}.

\subsection{Druhá podkapitola}
\hyplabel{dotextu}
Toto je moje druhá podkapitola. Sem povede odkaz z~textu.


%%% Uživatelská část.
\newpage
\section{Uživatelská část}
Toto je druhá část dokumentace a budu se zde zabývat tím, 
jak vypadá uživatelská část programu podle \cite{smith}.

\subsection{První podkapitola}\label{kapX}
Toto je text mé podkapitoly číslo \ref{kapX}

\begin{table}[ht]
  \begin{center}
    \renewcommand{\arraystretch}{1.2}
    \begin{tabular}{||l|rr||}
      \hline
      & \multicolumn{2}{|c||}{\bf \hbox{Informace}} \\
      \cline{2-3}
      \bf Čaj & \bf Cena & \bf Množství \\
      \hline
      Chun Mee & 30\,Kč & 100\,g \\
      Lung Ching & 86\,Kč & 50\,g \\
      Show Mee & 147\,Kč & 50\,g \\
      \hline
    \end{tabular}
    \caption{Toto je tabulka.} \label{tab}
  \end{center}
\end{table}

\subsection{Další podkapitola}
Toto je moje další podkapitola. Tato podkapitola bude dále členěna.

\subsubsection{Podkapitola}\label{podkapX}
Tato podkapitola má číslo \ref{podkapX} Zde budu řešit programování podle
knihy \cite{kovar}. V~této části je i~obrázek, viz \ref{obr}

\begin{figure}[ht]
  \centerline{\epsfbox{uplogo.eps}}
  \caption{Toto je obrázek.} \label{obr}
\end{figure}


%%% Programátorská část.
\newpage
\section{Programátorská část}
Toto je třetí část dokumentace a budu se zde zabývat tím, 
jak správně programovat podle knihy \cite{smith}.

\medskip
\suboutline{Vložená záložka}
Na toto místo směřuje záložka vytvořená pomocí makra \verb|\suboutline|.

\nextoutline{Jiné jméno}
\subsection{První podkapitola}\label{kapY}
Toto je text mé podkapitoly číslo \ref{kapY}
Tato kapitola se bude v~záložkách jmenovat jinak.

\subsection{Další podkapitola}
Toto je moje další podkapitola. Tato podkapitola bude dále členěna.

\subsubsection{Podkapitola}\label{podkapY}
Tato část má číslo~\ref{podkapY} a~je umístěna na stránce \pageref{podkapY}.
Na stránce uvedu i~odkaz na URL, třeba \url{http://www.inf.upol.cz}.
Dále lze vytvářet i~odkazy na kapitoly \emphref{v~rámci dokumentu}{kapX}.
Stejně tak lze vytvářet i~odkazy \emphref{přímo do textu}{dotextu},
pro vytvoření návěstí je potřeba použít makro \verb|\hyplabel|.

\medskip
Zde budu řešit programování podle knihy~\cite{kovar}.


\subsection{Poslední podkapitola}
A tady už budou jen závěrečné poznámky k~mému programování.


%%% Vytvoření seznamu literatury.
\newpage
\begin{thebibliography}{99}

\bibitem{smith} Smith, John. \emph{User and program.}
                Publisher, City, 1990.
\bibitem{kovar} Kovář, Jan. \emph{Jak programovat.}
                Nakladatelství, Město, 1990.
\bibitem{slozi} Složitý, Tomáš.
                \link{\emph{Překladač s~nakladačem.}}{http://www.inf.upol.cz}
                Elektronická publikace, 2001.

\end{thebibliography}

\end{document}
